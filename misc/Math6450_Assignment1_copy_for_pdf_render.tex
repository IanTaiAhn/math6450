\documentclass[11pt, twocolumn]{article}

    \usepackage[breakable]{tcolorbox}
    \usepackage{parskip} % Stop auto-indenting (to mimic markdown behaviour)
    

    % Basic figure setup, for now with no caption control since it's done
    % automatically by Pandoc (which extracts ![](path) syntax from Markdown).
    \usepackage{graphicx}
    % Keep aspect ratio if custom image width or height is specified
    \setkeys{Gin}{keepaspectratio}
    % Maintain compatibility with old templates. Remove in nbconvert 6.0
    \let\Oldincludegraphics\includegraphics
    % Ensure that by default, figures have no caption (until we provide a
    % proper Figure object with a Caption API and a way to capture that
    % in the conversion process - todo).
    \usepackage{caption}
    \DeclareCaptionFormat{nocaption}{}
    \captionsetup{format=nocaption,aboveskip=0pt,belowskip=0pt}

    \usepackage{float}
    \floatplacement{figure}{H} % forces figures to be placed at the correct location
    \usepackage{xcolor} % Allow colors to be defined
    \usepackage{enumerate} % Needed for markdown enumerations to work
    \usepackage{geometry} % Used to adjust the document margins
    \usepackage{amsmath} % Equations
    \usepackage{amssymb} % Equations
    \usepackage{textcomp} % defines textquotesingle
    % Hack from http://tex.stackexchange.com/a/47451/13684:
    \AtBeginDocument{%
        \def\PYZsq{\textquotesingle}% Upright quotes in Pygmentized code
    }
    \usepackage{upquote} % Upright quotes for verbatim code
    \usepackage{eurosym} % defines \euro

    \usepackage{iftex}
    \ifPDFTeX
        \usepackage[T1]{fontenc}
        \IfFileExists{alphabeta.sty}{
              \usepackage{alphabeta}
          }{
              \usepackage[mathletters]{ucs}
              \usepackage[utf8x]{inputenc}
          }
    \else
        \usepackage{fontspec}
        \usepackage{unicode-math}
    \fi

    % UNICODE CHARACTER SUPPORT - FIX FOR MISSING CHARACTERS
    % Add this section to handle Unicode characters in Verbatim environments
    \usepackage{newunicodechar}
    
    % Define replacements for problematic Unicode characters
    \newunicodechar{ρ}{\ensuremath{\rho}}
    \newunicodechar{≈}{\ensuremath{\approx}}
    \newunicodechar{≤}{\ensuremath{\leq}}
    \newunicodechar{≥}{\ensuremath{\geq}}
    \newunicodechar{×}{\ensuremath{\times}}
    \newunicodechar{β}{\ensuremath{\beta}}
    \newunicodechar{α}{\ensuremath{\alpha}}
    \newunicodechar{²}{\ensuremath{^2}}
    \newunicodechar{₀}{\ensuremath{_0}}
    \newunicodechar{₁}{\ensuremath{_1}}
    \newunicodechar{₂}{\ensuremath{_2}}
    % END UNICODE CHARACTER SUPPORT

    \usepackage{fancyvrb} % verbatim replacement that allows latex
    \usepackage{grffile} % extends the file name processing of package graphics
                         % to support a larger range
    \makeatletter % fix for old versions of grffile with XeLaTeX
    \@ifpackagelater{grffile}{2019/11/01}
    {
      % Do nothing on new versions
    }
    {
      \def\Gread@@xetex#1{%
        \IfFileExists{"\Gin@base".bb}%
        {\Gread@eps{\Gin@base.bb}}%
        {\Gread@@xetex@aux#1}%
      }
    }
    \makeatother
    \usepackage[Export]{adjustbox} % Used to constrain images to a maximum size
    \adjustboxset{max size={0.9\linewidth}{0.9\paperheight}}

    % The hyperref package gives us a pdf with properly built
    % internal navigation ('pdf bookmarks' for the table of contents,
    % internal cross-reference links, web links for URLs, etc.)
    \usepackage{hyperref}
    % The default LaTeX title has an obnoxious amount of whitespace. By default,
    % titling removes some of it. It also provides customization options.
    \usepackage{titling}
    \usepackage{longtable} % longtable support required by pandoc >1.10
    \usepackage{booktabs}  % table support for pandoc > 1.12.2
    \usepackage{array}     % table support for pandoc >= 2.11.3
    \usepackage{calc}      % table minipage width calculation for pandoc >= 2.11.1
    \usepackage[inline]{enumitem} % IRkernel/repr support (it uses the enumerate* environment)
    \usepackage[normalem]{ulem} % ulem is needed to support strikethroughs (\sout)
                                % normalem makes italics be italics, not underlines
    \usepackage{soul}      % strikethrough (\st) support for pandoc >= 3.0.0
    \usepackage{mathrsfs}
    

    
    % Colors for the hyperref package
    \definecolor{urlcolor}{rgb}{0,.145,.698}
    \definecolor{linkcolor}{rgb}{.71,0.21,0.01}
    \definecolor{citecolor}{rgb}{.12,.54,.11}

    % ANSI colors
    \definecolor{ansi-black}{HTML}{3E424D}
    \definecolor{ansi-black-intense}{HTML}{282C36}
    \definecolor{ansi-red}{HTML}{E75C58}
    \definecolor{ansi-red-intense}{HTML}{B22B31}
    \definecolor{ansi-green}{HTML}{00A250}
    \definecolor{ansi-green-intense}{HTML}{007427}
    \definecolor{ansi-yellow}{HTML}{DDB62B}
    \definecolor{ansi-yellow-intense}{HTML}{B27D12}
    \definecolor{ansi-blue}{HTML}{208FFB}
    \definecolor{ansi-blue-intense}{HTML}{0065CA}
    \definecolor{ansi-magenta}{HTML}{D160C4}
    \definecolor{ansi-magenta-intense}{HTML}{A03196}
    \definecolor{ansi-cyan}{HTML}{60C6C8}
    \definecolor{ansi-cyan-intense}{HTML}{258F8F}
    \definecolor{ansi-white}{HTML}{C5C1B4}
    \definecolor{ansi-white-intense}{HTML}{A1A6B2}
    \definecolor{ansi-default-inverse-fg}{HTML}{FFFFFF}
    \definecolor{ansi-default-inverse-bg}{HTML}{000000}

    % common color for the border for error outputs.
    \definecolor{outerrorbackground}{HTML}{FFDFDF}

    % commands and environments needed by pandoc snippets
    % extracted from the output of `pandoc -s`
    \providecommand{\tightlist}{%
      \setlength{\itemsep}{0pt}\setlength{\parskip}{0pt}}
    \DefineVerbatimEnvironment{Highlighting}{Verbatim}{commandchars=\\\{\}}
    % Add ',fontsize=\small' for more characters per line
    \newenvironment{Shaded}{}{}
    \newcommand{\KeywordTok}[1]{\textcolor[rgb]{0.00,0.44,0.13}{\textbf{{#1}}}}
    \newcommand{\DataTypeTok}[1]{\textcolor[rgb]{0.56,0.13,0.00}{{#1}}}
    \newcommand{\DecValTok}[1]{\textcolor[rgb]{0.25,0.63,0.44}{{#1}}}
    \newcommand{\BaseNTok}[1]{\textcolor[rgb]{0.25,0.63,0.44}{{#1}}}
    \newcommand{\FloatTok}[1]{\textcolor[rgb]{0.25,0.63,0.44}{{#1}}}
    \newcommand{\CharTok}[1]{\textcolor[rgb]{0.25,0.44,0.63}{{#1}}}
    \newcommand{\StringTok}[1]{\textcolor[rgb]{0.25,0.44,0.63}{{#1}}}
    \newcommand{\CommentTok}[1]{\textcolor[rgb]{0.38,0.63,0.69}{\textit{{#1}}}}
    \newcommand{\OtherTok}[1]{\textcolor[rgb]{0.00,0.44,0.13}{{#1}}}
    \newcommand{\AlertTok}[1]{\textcolor[rgb]{1.00,0.00,0.00}{\textbf{{#1}}}}
    \newcommand{\FunctionTok}[1]{\textcolor[rgb]{0.02,0.16,0.49}{{#1}}}
    \newcommand{\RegionMarkerTok}[1]{{#1}}
    \newcommand{\ErrorTok}[1]{\textcolor[rgb]{1.00,0.00,0.00}{\textbf{{#1}}}}
    \newcommand{\NormalTok}[1]{{#1}}

    % Additional commands for more recent versions of Pandoc
    \newcommand{\ConstantTok}[1]{\textcolor[rgb]{0.53,0.00,0.00}{{#1}}}
    \newcommand{\SpecialCharTok}[1]{\textcolor[rgb]{0.25,0.44,0.63}{{#1}}}
    \newcommand{\VerbatimStringTok}[1]{\textcolor[rgb]{0.25,0.44,0.63}{{#1}}}
    \newcommand{\SpecialStringTok}[1]{\textcolor[rgb]{0.73,0.40,0.53}{{#1}}}
    \newcommand{\ImportTok}[1]{{#1}}
    \newcommand{\DocumentationTok}[1]{\textcolor[rgb]{0.73,0.13,0.13}{\textit{{#1}}}}
    \newcommand{\AnnotationTok}[1]{\textcolor[rgb]{0.38,0.63,0.69}{\textbf{\textit{{#1}}}}}
    \newcommand{\CommentVarTok}[1]{\textcolor[rgb]{0.38,0.63,0.69}{\textbf{\textit{{#1}}}}}
    \newcommand{\VariableTok}[1]{\textcolor[rgb]{0.10,0.09,0.49}{{#1}}}
    \newcommand{\ControlFlowTok}[1]{\textcolor[rgb]{0.00,0.44,0.13}{\textbf{{#1}}}}
    \newcommand{\OperatorTok}[1]{\textcolor[rgb]{0.40,0.40,0.40}{{#1}}}
    \newcommand{\BuiltInTok}[1]{{#1}}
    \newcommand{\ExtensionTok}[1]{{#1}}
    \newcommand{\PreprocessorTok}[1]{\textcolor[rgb]{0.74,0.48,0.00}{{#1}}}
    \newcommand{\AttributeTok}[1]{\textcolor[rgb]{0.49,0.56,0.16}{{#1}}}
    \newcommand{\InformationTok}[1]{\textcolor[rgb]{0.38,0.63,0.69}{\textbf{\textit{{#1}}}}}
    \newcommand{\WarningTok}[1]{\textcolor[rgb]{0.38,0.63,0.69}{\textbf{\textit{{#1}}}}}
    \makeatletter
    \newsavebox\pandoc@box
    \newcommand*\pandocbounded[1]{%
      \sbox\pandoc@box{#1}%
      % scaling factors for width and height
      \Gscale@div\@tempa\textheight{\dimexpr\ht\pandoc@box+\dp\pandoc@box\relax}%
      \Gscale@div\@tempb\linewidth{\wd\pandoc@box}%
      % select the smaller of both
      \ifdim\@tempb\p@<\@tempa\p@
        \let\@tempa\@tempb
      \fi
      % scaling accordingly (\@tempa < 1)
      \ifdim\@tempa\p@<\p@
        \scalebox{\@tempa}{\usebox\pandoc@box}%
      % scaling not needed, use as it is
      \else
        \usebox{\pandoc@box}%
      \fi
    }
    \makeatother

    % Define a nice break command that doesn't care if a line doesn't already
    % exist.
    \def\br{\hspace*{\fill} \\* }
    % Math Jax compatibility definitions
    \def\gt{>}
    \def\lt{<}
    \let\Oldtex\TeX
    \let\Oldlatex\LaTeX
    \renewcommand{\TeX}{\textrm{\Oldtex}}
    \renewcommand{\LaTeX}{\textrm{\Oldlatex}}
    % Document parameters
    % Document title
    \title{Math6450\_Assignment1\_copy\_for\_pdf\_render}
    
    
% Pygments definitions
\makeatletter
\def\PY@reset{\let\PY@it=\relax \let\PY@bf=\relax%
    \let\PY@ul=\relax \let\PY@tc=\relax%
    \let\PY@bc=\relax \let\PY@ff=\relax}
\def\PY@tok#1{\csname PY@tok@#1\endcsname}
\def\PY@toks#1+{\ifx\relax#1\empty\else%
    \PY@tok{#1}\expandafter\PY@toks\fi}
\def\PY@do#1{\PY@bc{\PY@tc{\PY@ul{%
    \PY@it{\PY@bf{\PY@ff{#1}}}}}}}
\def\PY#1#2{\PY@reset\PY@toks#1+\relax+\PY@do{#2}}

\@namedef{PY@tok@w}{\def\PY@tc##1{\textcolor[rgb]{0.73,0.73,0.73}{##1}}}
\@namedef{PY@tok@c}{\let\PY@it=\textit\def\PY@tc##1{\textcolor[rgb]{0.24,0.48,0.48}{##1}}}
\@namedef{PY@tok@cp}{\def\PY@tc##1{\textcolor[rgb]{0.61,0.40,0.00}{##1}}}
\@namedef{PY@tok@k}{\let\PY@bf=\textbf\def\PY@tc##1{\textcolor[rgb]{0.00,0.50,0.00}{##1}}}
\@namedef{PY@tok@kp}{\def\PY@tc##1{\textcolor[rgb]{0.00,0.50,0.00}{##1}}}
\@namedef{PY@tok@kt}{\def\PY@tc##1{\textcolor[rgb]{0.69,0.00,0.25}{##1}}}
\@namedef{PY@tok@o}{\def\PY@tc##1{\textcolor[rgb]{0.40,0.40,0.40}{##1}}}
\@namedef{PY@tok@ow}{\let\PY@bf=\textbf\def\PY@tc##1{\textcolor[rgb]{0.67,0.13,1.00}{##1}}}
\@namedef{PY@tok@nb}{\def\PY@tc##1{\textcolor[rgb]{0.00,0.50,0.00}{##1}}}
\@namedef{PY@tok@nf}{\def\PY@tc##1{\textcolor[rgb]{0.00,0.00,1.00}{##1}}}
\@namedef{PY@tok@nc}{\let\PY@bf=\textbf\def\PY@tc##1{\textcolor[rgb]{0.00,0.00,1.00}{##1}}}
\@namedef{PY@tok@nn}{\let\PY@bf=\textbf\def\PY@tc##1{\textcolor[rgb]{0.00,0.00,1.00}{##1}}}
\@namedef{PY@tok@ne}{\let\PY@bf=\textbf\def\PY@tc##1{\textcolor[rgb]{0.80,0.25,0.22}{##1}}}
\@namedef{PY@tok@nv}{\def\PY@tc##1{\textcolor[rgb]{0.10,0.09,0.49}{##1}}}
\@namedef{PY@tok@no}{\def\PY@tc##1{\textcolor[rgb]{0.53,0.00,0.00}{##1}}}
\@namedef{PY@tok@nl}{\def\PY@tc##1{\textcolor[rgb]{0.46,0.46,0.00}{##1}}}
\@namedef{PY@tok@ni}{\let\PY@bf=\textbf\def\PY@tc##1{\textcolor[rgb]{0.44,0.44,0.44}{##1}}}
\@namedef{PY@tok@na}{\def\PY@tc##1{\textcolor[rgb]{0.41,0.47,0.13}{##1}}}
\@namedef{PY@tok@nt}{\let\PY@bf=\textbf\def\PY@tc##1{\textcolor[rgb]{0.00,0.50,0.00}{##1}}}
\@namedef{PY@tok@nd}{\def\PY@tc##1{\textcolor[rgb]{0.67,0.13,1.00}{##1}}}
\@namedef{PY@tok@s}{\def\PY@tc##1{\textcolor[rgb]{0.73,0.13,0.13}{##1}}}
\@namedef{PY@tok@sd}{\let\PY@it=\textit\def\PY@tc##1{\textcolor[rgb]{0.73,0.13,0.13}{##1}}}
\@namedef{PY@tok@si}{\let\PY@bf=\textbf\def\PY@tc##1{\textcolor[rgb]{0.64,0.35,0.47}{##1}}}
\@namedef{PY@tok@se}{\let\PY@bf=\textbf\def\PY@tc##1{\textcolor[rgb]{0.67,0.36,0.12}{##1}}}
\@namedef{PY@tok@sr}{\def\PY@tc##1{\textcolor[rgb]{0.64,0.35,0.47}{##1}}}
\@namedef{PY@tok@ss}{\def\PY@tc##1{\textcolor[rgb]{0.10,0.09,0.49}{##1}}}
\@namedef{PY@tok@sx}{\def\PY@tc##1{\textcolor[rgb]{0.00,0.50,0.00}{##1}}}
\@namedef{PY@tok@m}{\def\PY@tc##1{\textcolor[rgb]{0.40,0.40,0.40}{##1}}}
\@namedef{PY@tok@gh}{\let\PY@bf=\textbf\def\PY@tc##1{\textcolor[rgb]{0.00,0.00,0.50}{##1}}}
\@namedef{PY@tok@gu}{\let\PY@bf=\textbf\def\PY@tc##1{\textcolor[rgb]{0.50,0.00,0.50}{##1}}}
\@namedef{PY@tok@gd}{\def\PY@tc##1{\textcolor[rgb]{0.63,0.00,0.00}{##1}}}
\@namedef{PY@tok@gi}{\def\PY@tc##1{\textcolor[rgb]{0.00,0.52,0.00}{##1}}}
\@namedef{PY@tok@gr}{\def\PY@tc##1{\textcolor[rgb]{0.89,0.00,0.00}{##1}}}
\@namedef{PY@tok@ge}{\let\PY@it=\textit}
\@namedef{PY@tok@gs}{\let\PY@bf=\textbf}
\@namedef{PY@tok@ges}{\let\PY@bf=\textbf\let\PY@it=\textit}
\@namedef{PY@tok@gp}{\let\PY@bf=\textbf\def\PY@tc##1{\textcolor[rgb]{0.00,0.00,0.50}{##1}}}
\@namedef{PY@tok@go}{\def\PY@tc##1{\textcolor[rgb]{0.44,0.44,0.44}{##1}}}
\@namedef{PY@tok@gt}{\def\PY@tc##1{\textcolor[rgb]{0.00,0.27,0.87}{##1}}}
\@namedef{PY@tok@err}{\def\PY@bc##1{{\setlength{\fboxsep}{\string -\fboxrule}\fcolorbox[rgb]{1.00,0.00,0.00}{1,1,1}{\strut ##1}}}}
\@namedef{PY@tok@kc}{\let\PY@bf=\textbf\def\PY@tc##1{\textcolor[rgb]{0.00,0.50,0.00}{##1}}}
\@namedef{PY@tok@kd}{\let\PY@bf=\textbf\def\PY@tc##1{\textcolor[rgb]{0.00,0.50,0.00}{##1}}}
\@namedef{PY@tok@kn}{\let\PY@bf=\textbf\def\PY@tc##1{\textcolor[rgb]{0.00,0.50,0.00}{##1}}}
\@namedef{PY@tok@kr}{\let\PY@bf=\textbf\def\PY@tc##1{\textcolor[rgb]{0.00,0.50,0.00}{##1}}}
\@namedef{PY@tok@bp}{\def\PY@tc##1{\textcolor[rgb]{0.00,0.50,0.00}{##1}}}
\@namedef{PY@tok@fm}{\def\PY@tc##1{\textcolor[rgb]{0.00,0.00,1.00}{##1}}}
\@namedef{PY@tok@vc}{\def\PY@tc##1{\textcolor[rgb]{0.10,0.09,0.49}{##1}}}
\@namedef{PY@tok@vg}{\def\PY@tc##1{\textcolor[rgb]{0.10,0.09,0.49}{##1}}}
\@namedef{PY@tok@vi}{\def\PY@tc##1{\textcolor[rgb]{0.10,0.09,0.49}{##1}}}
\@namedef{PY@tok@vm}{\def\PY@tc##1{\textcolor[rgb]{0.10,0.09,0.49}{##1}}}
\@namedef{PY@tok@sa}{\def\PY@tc##1{\textcolor[rgb]{0.73,0.13,0.13}{##1}}}
\@namedef{PY@tok@sb}{\def\PY@tc##1{\textcolor[rgb]{0.73,0.13,0.13}{##1}}}
\@namedef{PY@tok@sc}{\def\PY@tc##1{\textcolor[rgb]{0.73,0.13,0.13}{##1}}}
\@namedef{PY@tok@dl}{\def\PY@tc##1{\textcolor[rgb]{0.73,0.13,0.13}{##1}}}
\@namedef{PY@tok@s2}{\def\PY@tc##1{\textcolor[rgb]{0.73,0.13,0.13}{##1}}}
\@namedef{PY@tok@sh}{\def\PY@tc##1{\textcolor[rgb]{0.73,0.13,0.13}{##1}}}
\@namedef{PY@tok@s1}{\def\PY@tc##1{\textcolor[rgb]{0.73,0.13,0.13}{##1}}}
\@namedef{PY@tok@mb}{\def\PY@tc##1{\textcolor[rgb]{0.40,0.40,0.40}{##1}}}
\@namedef{PY@tok@mf}{\def\PY@tc##1{\textcolor[rgb]{0.40,0.40,0.40}{##1}}}
\@namedef{PY@tok@mh}{\def\PY@tc##1{\textcolor[rgb]{0.40,0.40,0.40}{##1}}}
\@namedef{PY@tok@mi}{\def\PY@tc##1{\textcolor[rgb]{0.40,0.40,0.40}{##1}}}
\@namedef{PY@tok@il}{\def\PY@tc##1{\textcolor[rgb]{0.40,0.40,0.40}{##1}}}
\@namedef{PY@tok@mo}{\def\PY@tc##1{\textcolor[rgb]{0.40,0.40,0.40}{##1}}}
\@namedef{PY@tok@ch}{\let\PY@it=\textit\def\PY@tc##1{\textcolor[rgb]{0.24,0.48,0.48}{##1}}}
\@namedef{PY@tok@cm}{\let\PY@it=\textit\def\PY@tc##1{\textcolor[rgb]{0.24,0.48,0.48}{##1}}}
\@namedef{PY@tok@cpf}{\let\PY@it=\textit\def\PY@tc##1{\textcolor[rgb]{0.24,0.48,0.48}{##1}}}
\@namedef{PY@tok@c1}{\let\PY@it=\textit\def\PY@tc##1{\textcolor[rgb]{0.24,0.48,0.48}{##1}}}
\@namedef{PY@tok@cs}{\let\PY@it=\textit\def\PY@tc##1{\textcolor[rgb]{0.24,0.48,0.48}{##1}}}

\def\PYZbs{\char`\\}
\def\PYZus{\char`\_}
\def\PYZob{\char`\{}
\def\PYZcb{\char`\}}
\def\PYZca{\char`\^}
\def\PYZam{\char`\&}
\def\PYZlt{\char`\<}
\def\PYZgt{\char`\>}
\def\PYZsh{\char`\#}
\def\PYZpc{\char`\%}
\def\PYZdl{\char`\$}
\def\PYZhy{\char`\-}
\def\PYZsq{\char`\'}
\def\PYZdq{\char`\"}
\def\PYZti{\char`\~}
% for compatibility with earlier versions
\def\PYZat{@}
\def\PYZlb{[}
\def\PYZrb{]}
\makeatother


    % For linebreaks inside Verbatim environment from package fancyvrb.
    \makeatletter
        \newbox\Wrappedcontinuationbox
        \newbox\Wrappedvisiblespacebox
        \newcommand*\Wrappedvisiblespace {\textcolor{red}{\textvisiblespace}}
        \newcommand*\Wrappedcontinuationsymbol {\textcolor{red}{\llap{\tiny$\m@th\hookrightarrow$}}}
        \newcommand*\Wrappedcontinuationindent {3ex }
        \newcommand*\Wrappedafterbreak {\kern\Wrappedcontinuationindent\copy\Wrappedcontinuationbox}
        % Take advantage of the already applied Pygments mark-up to insert
        % potential linebreaks for TeX processing.
        %        {, <, #, %, $, ' and ": go to next line.
        %        _, }, ^, &, >, - and ~: stay at end of broken line.
        % Use of \textquotesingle for straight quote.
        \newcommand*\Wrappedbreaksatspecials {%
            \def\PYGZus{\discretionary{\char`\_}{\Wrappedafterbreak}{\char`\_}}%
            \def\PYGZob{\discretionary{}{\Wrappedafterbreak\char`\{}{\char`\{}}%
            \def\PYGZcb{\discretionary{\char`\}}{\Wrappedafterbreak}{\char`\}}}%
            \def\PYGZca{\discretionary{\char`\^}{\Wrappedafterbreak}{\char`\^}}%
            \def\PYGZam{\discretionary{\char`\&}{\Wrappedafterbreak}{\char`\&}}%
            \def\PYGZlt{\discretionary{}{\Wrappedafterbreak\char`\<}{\char`\<}}%
            \def\PYGZgt{\discretionary{\char`\>}{\Wrappedafterbreak}{\char`\>}}%
            \def\PYGZsh{\discretionary{}{\Wrappedafterbreak\char`\#}{\char`\#}}%
            \def\PYGZpc{\discretionary{}{\Wrappedafterbreak\char`\%}{\char`\%}}%
            \def\PYGZdl{\discretionary{}{\Wrappedafterbreak\char`\$}{\char`\$}}%
            \def\PYGZhy{\discretionary{\char`\-}{\Wrappedafterbreak}{\char`\-}}%
            \def\PYGZsq{\discretionary{}{\Wrappedafterbreak\textquotesingle}{\textquotesingle}}%
            \def\PYGZdq{\discretionary{}{\Wrappedafterbreak\char`\"}{\char`\"}}%
            \def\PYGZti{\discretionary{\char`\~}{\Wrappedafterbreak}{\char`\~}}%
        }
        % Some characters . , ; ? ! / are not pygmentized.
        % This macro makes them "active" and they will insert potential linebreaks
        \newcommand*\Wrappedbreaksatpunct {%
            \lccode`\~`\.\lowercase{\def~}{\discretionary{\hbox{\char`\.}}{\Wrappedafterbreak}{\hbox{\char`\.}}}%
            \lccode`\~`\,\lowercase{\def~}{\discretionary{\hbox{\char`\,}}{\Wrappedafterbreak}{\hbox{\char`\,}}}%
            \lccode`\~`\;\lowercase{\def~}{\discretionary{\hbox{\char`\;}}{\Wrappedafterbreak}{\hbox{\char`\;}}}%
            \lccode`\~`\:\lowercase{\def~}{\discretionary{\hbox{\char`\:}}{\Wrappedafterbreak}{\hbox{\char`\:}}}%
            \lccode`\~`\?\lowercase{\def~}{\discretionary{\hbox{\char`\?}}{\Wrappedafterbreak}{\hbox{\char`\?}}}%
            \lccode`\~`\!\lowercase{\def~}{\discretionary{\hbox{\char`\!}}{\Wrappedafterbreak}{\hbox{\char`\!}}}%
            \lccode`\~`\/\lowercase{\def~}{\discretionary{\hbox{\char`\/}}{\Wrappedafterbreak}{\hbox{\char`\/}}}%
            \catcode`\.\active
            \catcode`\,\active
            \catcode`\;\active
            \catcode`\:\active
            \catcode`\?\active
            \catcode`\!\active
            \catcode`\/\active
            \lccode`\~`\~
        }
    \makeatother

    \let\OriginalVerbatim=\Verbatim
    \makeatletter
    \renewcommand{\Verbatim}[1][1]{%
        %\parskip\z@skip
        \sbox\Wrappedcontinuationbox {\Wrappedcontinuationsymbol}%
        \sbox\Wrappedvisiblespacebox {\FV@SetupFont\Wrappedvisiblespace}%
        \def\FancyVerbFormatLine ##1{\hsize\linewidth
            \vtop{\raggedright\hyphenpenalty\z@\exhyphenpenalty\z@
                \doublehyphendemerits\z@\finalhyphendemerits\z@
                \strut ##1\strut}%
        }%
        % If the linebreak is at a space, the latter will be displayed as visible
        % space at end of first line, and a continuation symbol starts next line.
        % Stretch/shrink are however usually zero for typewriter font.
        \def\FV@Space {%
            \nobreak\hskip\z@ plus\fontdimen3\font minus\fontdimen4\font
            \discretionary{\copy\Wrappedvisiblespacebox}{\Wrappedafterbreak}
            {\kern\fontdimen2\font}%
        }%

        % Allow breaks at special characters using \PYG... macros.
        \Wrappedbreaksatspecials
        % Breaks at punctuation characters . , ; ? ! and / need catcode=\active
        \OriginalVerbatim[#1,codes*=\Wrappedbreaksatpunct]%
    }
    \makeatother

    % Exact colors from NB
    \definecolor{incolor}{HTML}{303F9F}
    \definecolor{outcolor}{HTML}{D84315}
    \definecolor{cellborder}{HTML}{CFCFCF}
    \definecolor{cellbackground}{HTML}{F7F7F7}

    % prompt
    \makeatletter
    \newcommand{\boxspacing}{\kern\kvtcb@left@rule\kern\kvtcb@boxsep}
    \makeatother
    \newcommand{\prompt}[4]{
        {\ttfamily\llap{{\color{#2}[#3]:\hspace{3pt}#4}}\vspace{-\baselineskip}}
    }
    

    
    % Prevent overflowing lines due to hard-to-break entities
    \sloppy
    % Setup hyperref package
    \hypersetup{
      breaklinks=true,  % so long urls are correctly broken across lines
      colorlinks=true,
      urlcolor=urlcolor,
      linkcolor=linkcolor,
      citecolor=citecolor,
      }
    % Slightly bigger margins than the latex defaults
    
    \geometry{verbose,tmargin=1in,bmargin=1in,lmargin=1in,rmargin=1in}
    
    

\begin{document}
    
    \maketitle
    
    

    
    Part 1: Data Exploration and Preparation

    \begin{Verbatim}[commandchars=\\\{\}]
============================================================
BOSTON HOUSING DATASET ANALYSIS
============================================================

1.1 DATASET DIMENSIONS
------------------------------
Number of observations (rows): 506
Number of variables (columns): 14
Dataset shape: (506, 14)

Column names: ['crim', 'zn', 'indus', 'chas', 'nox', 'rm', 'age', 'dis', 'rad',
'tax', 'ptratio', 'b', 'lstat', 'medv']
    \end{Verbatim}

    \begin{Verbatim}[commandchars=\\\{\}]

1.2 DESCRIPTIVE STATISTICS
------------------------------

Descriptive statistics for TARGET VARIABLE (medv):
count   506.000
mean     22.533
std       9.197
min       5.000
25\%      17.025
50\%      21.200
75\%      25.000
max      50.000
Name: medv, dtype: float64

Descriptive statistics for PRIMARY FEATURE (lstat):
count   506.000
mean     12.653
std       7.141
min       1.730
25\%       6.950
50\%      11.360
75\%      16.955
max      37.970
Name: lstat, dtype: float64

Additional statistics for medv:
Variance: 84.5867
Standard deviation: 9.1971
Skewness: 1.1081
Kurtosis: 1.4952

Additional statistics for lstat:
Variance: 50.9948
Standard deviation: 7.1411
Skewness: 0.9065
Kurtosis: 0.4932
    \end{Verbatim}

    \begin{Verbatim}[commandchars=\\\{\}]

1.3 CORRELATION ANALYSIS
------------------------------
Correlation coefficient between medv and lstat: -0.7377

INTERPRETA1TION:
- The correlation coefficient of -0.7377 indicates a strong negative
relationship
- This means that as lstat (\% lower status population) increases, medv (median
home value) tends to decrease
- The relationship explains approximately 54.4\% of the variance (R² = 0.5441)
- Statistical significance: p-value = 5.08e-88
- The correlation is statistically significant at α = 0.05
    \end{Verbatim}

    \begin{Verbatim}[commandchars=\\\{\}]

1.4 SCATTER PLOT ANALYSIS
------------------------------
    \end{Verbatim}

    \begin{center}
    \adjustimage{max size={0.9\linewidth}{0.9\paperheight}}{Math6450_Assignment1_copy_for_pdf_render_files/Math6450_Assignment1_copy_for_pdf_render_4_1.png}
    \end{center}
    { \hspace*{\fill} \\}
    
    \begin{Verbatim}[commandchars=\\\{\}]
PATTERN OBSERVED IN SCATTER PLOT:
- The scatter plot reveals a clear negative relationship between lstat and medv
- As the percentage of lower status population increases, median home values
tend to decrease
- The relationship appears to be non-linear, showing a curved pattern rather
than a straight line
- There's more variability in home values at lower lstat percentages
- The relationship seems stronger (steeper decline) at lower lstat values and
levels off at higher lstat values
- There are some potential outliers, particularly homes with high values despite
higher lstat percentages
- The data points form a characteristic negative exponential or power-law
pattern
    \end{Verbatim}

    \begin{center}
    \adjustimage{max size={0.9\linewidth}{0.9\paperheight}}{Math6450_Assignment1_copy_for_pdf_render_files/Math6450_Assignment1_copy_for_pdf_render_4_3.png}
    \end{center}
    { \hspace*{\fill} \\}
    
    \begin{Verbatim}[commandchars=\\\{\}]

SUMMARY:
- Dataset contains 506 observations and 14 variables
- Strong negative correlation (-0.7377) between lstat and medv
- Non-linear relationship visible in scatter plot
- Both variables show reasonable distributions for regression analysis
    \end{Verbatim}

    Part 2: Linear Regression Model Fitting

    2.1 Write the estimated regression equation in the form:

\(\text{medv} = \hat{\beta}_0 + \hat{\beta}_1 \times \text{lstat}\)

\[
\text{medv} = \hat{\beta}_0 + \hat{\beta}_1 \times \text{lstat}
\]

    \begin{Verbatim}[commandchars=\\\{\}]

COEFFICIENTS:
Intercept (β₀): 34.5538
Slope (β₁): -0.9500

2.1 ESTIMATED REGRESSION EQUATION
----------------------------------------
medv = 34.5538 + (-0.9500) × lstat
medv = 34.5538 - 0.9500 × lstat

Alternative notation:
ŷ = 34.5538 + (-0.9500)x
where ŷ = predicted median home value and x = lstat
    \end{Verbatim}

    \begin{Verbatim}[commandchars=\\\{\}]

2.2 INTERPRETATION OF INTERCEPT (β₀)
----------------------------------------
Intercept value: 34.5538

INTERPRETATION:
- The intercept represents the predicted median home value when lstat = 0
- This means when 0\% of the population has lower status, the predicted median
home value is \$34.55k
- In practical terms: \$34554

PRACTICAL MEANING:
- Observed lstat range: 1.73\% to 37.97\%
- Since the minimum observed lstat is 1.73\%, lstat = 0 is outside our data range
- Therefore, the intercept represents extrapolation beyond observed data
- While mathematically meaningful, it has LIMITED PRACTICAL MEANING because:
  * No area in the dataset has 0\% lower status population
  * Real-world interpretation: represents the 'theoretical maximum' home value
  * Should be interpreted cautiously due to extrapolation
    \end{Verbatim}

    \begin{Verbatim}[commandchars=\\\{\}]

2.3 INTERPRETATION OF SLOPE (β₁)
----------------------------------------
Slope value: -0.9500

INTERPRETATION:
For each 1\% increase in lstat (lower status population), the median home value
decreases by \$0.9500k on average, holding all other factors constant.

In practical terms:
- A 1\% increase in lower status population is associated with a \$950 decrease in
median home value
- A 5\% increase in lower status population would decrease median home value by
\$4750
- A 10\% increase in lower status population would decrease median home value by
\$9500
    \end{Verbatim}

    2.4 Based on the 95\% confidence intervals for the coefficients, are
both the intercept and slope significantly different from zero? Support
your answer with the confidence interval values.

todo

    \begin{Verbatim}[commandchars=\\\{\}]

2.4 CONFIDENCE INTERVALS AND SIGNIFICANCE TESTING
--------------------------------------------------
95\% CONFIDENCE INTERVALS:
               0      1
Intercept 33.448 35.659
lstat     -1.026 -0.874

DETAILED CONFIDENCE INTERVALS:
Intercept (β₀): [33.4485, 35.6592]
Slope (β₁): [-1.0261, -0.8740]

SIGNIFICANCE TESTING:
H₀: β = 0 (coefficient equals zero)
H₁: β ≠ 0 (coefficient is significantly different from zero)

INTERCEPT (β₀) ANALYSIS:
- 95\% CI: [33.4485, 35.6592]
- Contains zero? No
- Conclusion: The intercept IS significantly different from zero
- This means we can be 95\% confident the true intercept is between 33.4485 and
35.6592

SLOPE (β₁) ANALYSIS:
- 95\% CI: [-1.0261, -0.8740]
- Contains zero? No
- Conclusion: The slope IS significantly different from zero
- This means we can be 95\% confident the true slope is between -1.0261 and
-0.8740

P-VALUES (for additional confirmation):
Intercept p-value: 3.74e-236
Slope p-value: 5.08e-88
Both p-values < 0.05: True

MODEL SUMMARY STATISTICS:
R-squared: 0.5441
Adjusted R-squared: 0.5432
F-statistic: 601.62
F-statistic p-value: 5.08e-88
Standard Error: 6.2158
    \end{Verbatim}

    \begin{center}
    \adjustimage{max size={0.9\linewidth}{0.9\paperheight}}{Math6450_Assignment1_copy_for_pdf_render_files/Math6450_Assignment1_copy_for_pdf_render_11_1.png}
    \end{center}
    { \hspace*{\fill} \\}
    
    \begin{Verbatim}[commandchars=\\\{\}]

FINAL SUMMARY:
- Regression equation: medv = 34.5538 + (-0.9500) × lstat
- Both coefficients are statistically significant at α = 0.05
- The model explains 54.4\% of the variance in median home values
- For every 1\% increase in lower status population, median home value decreases
by \$950 on average
    \end{Verbatim}

    Task 2.2: Model Performance Evaluation

    2.5 What is the R-squared value? Interpret this in terms of the
percentage of variation in median home values explained by the
percentage of lower status population.

todo

    \begin{Verbatim}[commandchars=\\\{\}]

2.5 R-SQUARED ANALYSIS
------------------------------
R-squared value: 0.5441
R-squared as percentage: 54.41\%

INTERPRETATION:
- R² = 0.5441 means that 54.41\% of the variation in median home values
  is explained by the percentage of lower status population (lstat)
- The remaining 45.59\% of variation is due to other factors not included in this
model
- This indicates a moderate relationship
- In practical terms: knowing the lstat value allows us to predict about 54.4\%
of the variation in home values
    \end{Verbatim}

    \begin{Verbatim}[commandchars=\\\{\}]

2.6 ROOT MEAN SQUARE ERROR (RMSE)
-----------------------------------
Mean Squared Error (MSE): 38.6357
Root Mean Square Error (RMSE): 6.2158

INTERPRETATION:
- RMSE = 6.2158 thousands of dollars
- In actual dollars: \$6216
- This means the typical prediction error is approximately \$6216
- On average, our predictions are off by about ±\$6216 from the actual median
home value

CONTEXT:
- Mean home value: \$22.53k (\$22533)
- Standard deviation of home values: \$9.20k
- Range of home values: \$45.00k
- RMSE as \% of mean: 27.6\%
- RMSE as \% of standard deviation: 67.6\%
    \end{Verbatim}

    \begin{Verbatim}[commandchars=\\\{\}]

2.7 F-STATISTIC AND OVERALL MODEL SIGNIFICANCE
---------------------------------------------
F-statistic: 601.6179
F-statistic p-value: 5.08e-88
Degrees of freedom: Model = 1.0, Residual = 504.0

HYPOTHESIS TEST:
H₀: The model has no explanatory power (β₁ = 0)
H₁: The model has explanatory power (β₁ ≠ 0)

INTERPRETATION:
- F-statistic = 601.6179 with p-value = 5.08e-88
- Since p-value < 0.05, we REJECT the null hypothesis
- Conclusion: The model IS statistically significant
- This means lstat DOES have significant explanatory power for predicting medv

PRACTICAL MEANING:
- The F-test confirms that our regression model performs significantly better
  than a model with no predictors (just the mean)
- The relationship between lstat and medv is statistically meaningful
- We can be confident that lstat is a useful predictor of median home values
    \end{Verbatim}

    2.8 Compare the adjusted R-squared with the regular R-squared. Why might
there be a difference, and what does the adjusted version account for?
todo

    \begin{Verbatim}[commandchars=\\\{\}]

2.8 ADJUSTED R-SQUARED COMPARISON
-----------------------------------
R-squared: 0.544146
Adjusted R-squared: 0.543242
Difference: 0.000904

WHY THERE MIGHT BE A DIFFERENCE:
- Regular R²: 0.544146
- Adjusted R²: 0.543242
- The difference of 0.000904 is very small

WHAT ADJUSTED R-SQUARED ACCOUNTS FOR:
- Number of predictors in the model: 1.0
- Sample size: 506 observations
- Degrees of freedom penalty for adding predictors

FORMULA EXPLANATION:
Adjusted R² = 1 - [(1 - R²) × (n - 1) / (n - k - 1)]
where n = sample size (506) and k = number of predictors (1.0)
Manual calculation: 0.543242

INTERPRETATION:
- The very small difference suggests our model is not overfitting
- With only one predictor, the adjustment is minimal
- Both R² and adjusted R² tell essentially the same story

PRACTICAL IMPLICATIONS:
- For model comparison: Use adjusted R² when comparing models with different
numbers of predictors
- For interpretation: Both values are nearly identical, indicating a robust
single-predictor model
- The penalty for our one predictor is minimal given the sample size of 506
observations

FINAL SUMMARY:
==================================================
- R² = 0.5441 (54.41\% of variance explained)
- Adjusted R² = 0.5432 (54.32\% of variance explained)
- RMSE = \$6216 (typical prediction error)
- F-statistic = 601.6179, p < 0.05 (highly significant model)
- Model explains 54.4\% of home value variation using just lstat
- Typical prediction accuracy: ±\$6216 (27.6\% of mean home value)
    \end{Verbatim}

    Part 3: Statistical Inference and Hypothesis Testing

    \begin{Verbatim}[commandchars=\\\{\}]

3.1 HYPOTHESIS TESTING SETUP
----------------------------------------
TESTING THE SLOPE COEFFICIENT:
H₀: β₁ = 0 (The slope coefficient is zero)
    → lstat has no linear relationship with medv
    → There is no linear association between \% lower status population and
median home value

H₁: β₁ ≠ 0 (The slope coefficient is not zero)
    → lstat has a significant linear relationship with medv
    → There is a significant linear association between \% lower status
population and median home value

Type of test: Two-tailed test
Significance level: α = 0.05
    \end{Verbatim}

    \begin{Verbatim}[commandchars=\\\{\}]

3.2 T-STATISTIC AND P-VALUE ANALYSIS
---------------------------------------------
TEST STATISTICS:
t-statistic: -24.527900
p-value: 5.08e-88
Degrees of freedom: 504.0
Critical t-value (α = 0.05, two-tailed): ±1.9647

DECISION MAKING:
Decision rule: Reject H₀ if |t| > 1.9647 OR if p-value < 0.05
Observed: |t| = 24.5279, p-value = 5.08e-88

CONCLUSION AT 5\% SIGNIFICANCE LEVEL:
✓ REJECT H₀: The slope coefficient IS significantly different from zero
  - |t| = 24.5279 > 1.9647 ✓
  - p-value = 5.08e-88 < 0.05 ✓
  - Statistical evidence: There IS a significant linear relationship between
lstat and medv

PRACTICAL INTERPRETATION:
- We can be 95\% confident that changes in \% lower status population
  have a real, measurable effect on median home values
- The relationship observed in our sample is unlikely to be due to random chance
- The effect size: each 1\% increase in lstat is associated with
  a \$950 decrease in median home value
    \end{Verbatim}

    3.3 Calculate and interpret the 99\% confidence interval for the slope
coefficient. How does this compare to the 95\% interval in terms of
width and interpretation?

todo

    \begin{Verbatim}[commandchars=\\\{\}]

3.3 CONFIDENCE INTERVAL ANALYSIS
----------------------------------------
CONFIDENCE INTERVALS FOR SLOPE COEFFICIENT:
95\% Confidence Interval: [-1.026148, -0.873951]
99\% Confidence Interval: [-1.050199, -0.849899]

INTERVAL WIDTH COMPARISON:
95\% CI width: 0.152198
99\% CI width: 0.200300
Width increase: 0.048102
Percent increase in width: 31.6\%

INTERPRETATION:
95\% CONFIDENCE INTERVAL:
- We are 95\% confident that the true slope coefficient lies between
  -1.026148 and -0.873951
- In practical terms: each 1\% increase in lstat decreases median home value
  by between \$874 and \$1026

99\% CONFIDENCE INTERVAL:
- We are 99\% confident that the true slope coefficient lies between
  -1.050199 and -0.849899
- In practical terms: each 1\% increase in lstat decreases median home value
  by between \$850 and \$1050

COMPARISON ANALYSIS:
- The 99\% CI is wider than the 95\% CI by 0.048102
- This represents a 31.6\% increase in width
- WHY: Higher confidence level requires a wider interval to capture the true
parameter
- TRADE-OFF: More confidence (99\% vs 95\%) comes at the cost of precision (wider
interval)

SIGNIFICANCE IMPLICATIONS:
95\% CI contains zero: No
99\% CI contains zero: No
- Since neither interval contains zero, the slope is significant at both levels
- This provides strong evidence for a real relationship between lstat and medv
    \end{Verbatim}

    3.4 If someone claimed that each 1\% increase in lstat decreases median
home value by exactly \$1000, would your regression results support or
contradict this claim? Justify your answer using statistical evidence.

todo

    \begin{Verbatim}[commandchars=\\\{\}]

3.4 TESTING SPECIFIC CLAIM
-----------------------------------
CLAIM TO TEST:
Someone claims that each 1\% increase in lstat decreases median home value by
exactly \$1000
In our units: β₁ = -1.0 (since medv is in thousands of dollars)

HYPOTHESES:
H₀: β₁ = -1.0 (the claim is correct)
H₁: β₁ ≠ -1.0 (the claim is incorrect)

TEST USING CONFIDENCE INTERVALS:
Observed slope coefficient: -0.950049
Claimed slope coefficient: -1.0

95\% Confidence Interval Test:
- 95\% CI: [-1.026148, -0.873951]
- Does the CI contain -1.0? Yes

99\% Confidence Interval Test:
- 99\% CI: [-1.050199, -0.849899]
- Does the CI contain -1.0? Yes

FORMAL T-TEST:
t-statistic = (observed - claimed) / SE = (-0.950049 - -1.0) / 0.038733
t-statistic = 1.2896
p-value (two-tailed): 0.1978

CONCLUSION:
✓ FAIL TO REJECT the claim at 95\% confidence level
  - The claimed value (-1.0) IS within the 95\% confidence interval
  - Our regression results SUPPORT the claim
✓ FAIL TO REJECT the claim at 99\% confidence level
  - The claimed value (-1.0) IS within the 99\% confidence interval

STATISTICAL EVIDENCE:
- Our estimate: Each 1\% increase in lstat decreases home value by \$950
- Claimed effect: Each 1\% increase in lstat decreases home value by \$1000
- Difference: \$50
- The difference is not statistically significant (p = 0.1978 ≥ 0.05)
- Insufficient evidence to reject the claim
    \end{Verbatim}

    \begin{center}
    \adjustimage{max size={0.9\linewidth}{0.9\paperheight}}{Math6450_Assignment1_copy_for_pdf_render_files/Math6450_Assignment1_copy_for_pdf_render_26_1.png}
    \end{center}
    { \hspace*{\fill} \\}
    
    \begin{Verbatim}[commandchars=\\\{\}]

FINAL SUMMARY:
============================================================
3.1 Hypotheses: H₀: β₁ = 0 vs H₁: β₁ ≠ 0
3.2 Test results: t = -24.5279, p = 5.08e-88
    Conclusion: Reject H₀ - slope is significant
3.3 Confidence intervals:
    95\% CI: [-1.026148, -0.873951] (width: 0.152198)
    99\% CI: [-1.050199, -0.849899] (width: 0.200300)
    99\% CI is 31.6\% wider than 95\% CI
3.4 Claim test: The claim of exactly \$1000 decrease is SUPPORTED
    Our estimate: \$950 decrease per 1\% lstat increase
    Statistical significance of difference: p = 0.1978
    \end{Verbatim}

    Part 4: Assumption Testing and Model Diagnostics

    4.1 Perform the Shapiro-Wilk test for normality of residuals. Report the
test statistic, p-value, and your conclusion at the 5\% significance
level.

todo

    \begin{Verbatim}[commandchars=\\\{\}]
================================================================================
BOSTON HOUSING ASSUMPTION TESTING AND MODEL DIAGNOSTICS - PART 4
================================================================================
MODEL SUMMARY:
Sample size: 506
Number of residuals: 506
Mean of residuals: 0.000000 (should be ≈ 0)
Standard deviation of residuals: 6.2096

4.1 SHAPIRO-WILK TEST FOR NORMALITY OF RESIDUALS
-------------------------------------------------------
HYPOTHESIS TESTING:
H₀: Residuals follow a normal distribution
H₁: Residuals do not follow a normal distribution
Significance level: α = 0.05

TEST RESULTS:
Shapiro-Wilk test statistic (W): 0.878572
p-value: 0.000000

DECISION MAKING:
Decision rule: Reject H₀ if p-value < 0.05
Observed p-value: 0.000000

CONCLUSION AT 5\% SIGNIFICANCE LEVEL:
✗ REJECT H₀: Residuals do not follow a normal distribution
  - Statistical evidence suggests departure from normality
  - The normality assumption may be violated

INTERPRETation OF TEST STATISTIC:
- W = 0.878572
- W ranges from 0 to 1, with values closer to 1 indicating more normal-like data
- Our value suggests weak evidence of normality based on the test statistic
alone

ADDITIONAL NORMALITY TESTS (for comparison):
D'Agostino's test: statistic = 137.0434, p-value = 0.000000
Jarque-Bera test: statistic = 291.3734, p-value = 0.000000

CONSENSUS: Tests show mixed results regarding normality
    \end{Verbatim}

    4.2 Create a Q-Q plot of the residuals. Does the visual evidence support
or contradict your statistical test result? Explain what you observe.

    \begin{Verbatim}[commandchars=\\\{\}]

4.2 Q-Q PLOT ANALYSIS
-------------------------
Q-Q PLOT INTERPRETATION:
The Q-Q (Quantile-Quantile) plot compares residual quantiles to theoretical
normal quantiles
Q-Q plot correlation: 0.9373
(Values closer to 1 indicate better fit to normal distribution)

VISUAL ASSESSMENT:
- Good fit with minor deviations
- Look for points following the red diagonal line
- Systematic deviations suggest non-normality
    \end{Verbatim}

    \begin{center}
    \adjustimage{max size={0.9\linewidth}{0.9\paperheight}}{Math6450_Assignment1_copy_for_pdf_render_files/Math6450_Assignment1_copy_for_pdf_render_31_1.png}
    \end{center}
    { \hspace*{\fill} \\}
    
    4.3 Create a histogram of residuals with a normal distribution overlay.
Comment on the shape of the distribution and any departures from
normality.

todo

    \begin{Verbatim}[commandchars=\\\{\}]

4.3 HISTOGRAM WITH NORMAL DISTRIBUTION OVERLAY
--------------------------------------------------
SHAPE ANALYSIS:
Skewness: 1.4527
Kurtosis: 2.3191 (excess kurtosis)

SKEWNESS INTERPRETATION:
- Skewness = 1.4527 indicates highly skewed
- Distribution is skewed to the right

KURTOSIS INTERPRETATION:
- Excess kurtosis = 2.3191 indicates heavy-tailed (leptokurtic)
- Normal distribution has excess kurtosis = 0
    \end{Verbatim}

    \begin{center}
    \adjustimage{max size={0.9\linewidth}{0.9\paperheight}}{Math6450_Assignment1_copy_for_pdf_render_files/Math6450_Assignment1_copy_for_pdf_render_33_1.png}
    \end{center}
    { \hspace*{\fill} \\}
    
    \begin{Verbatim}[commandchars=\\\{\}]

DEPARTURES FROM NORMALITY:
Identified departures from normality:
1. Skewness (1.453)
2. Kurtosis (2.319)
3. Shapiro-Wilk test rejection
4. Q-Q plot deviations

4.2 VISUAL EVIDENCE VS STATISTICAL TEST COMPARISON:
--------------------------------------------------
Statistical test result (Shapiro-Wilk): Rejects normality
Visual evidence assessment: Shows deviations from normality
✓ AGREEMENT: Visual evidence and statistical test both suggest departure from
normality

DETAILED VISUAL OBSERVATIONS:
Q-Q Plot:
- Systematic deviations from diagonal line (r = 0.9373)
- Visual evidence against perfect normality

Histogram:
- Notable departures from bell-shaped normal distribution
- Skewness and/or kurtosis concerns visible

PRACTICAL IMPLICATIONS FOR REGRESSION:
----------------------------------------
⚠ NORMALITY ASSUMPTION VIOLATED:
  - Confidence intervals may be less reliable
  - Consider robust standard errors
  - Prediction intervals may be inaccurate
  - Consider variable transformation

SAMPLE SIZE CONSIDERATIONS:
- Sample size: 506 observations
- Large sample: Central Limit Theorem helps with normality concerns
- Minor deviations from normality are less problematic

FINAL SUMMARY:
============================================================
4.1 Shapiro-Wilk test: W = 0.878572, p = 0.000000
    Conclusion: Residuals deviate from normality
4.2 Q-Q plot assessment: r = 0.9373
    Visual evidence: Shows deviations from normality
4.3 Histogram analysis:
    Skewness: 1.4527, Kurtosis: 2.3191
    Shape: highly skewed, heavy-tailed (leptokurtic)
Overall normality assessment: VIOLATED
    \end{Verbatim}

    Task 4.2: Homoscedasticity Testing

    4.4 Perform the Breusch-Pagan test for homoscedasticity. Report the test
statistic, p-value, and your conclusion.

    \begin{Verbatim}[commandchars=\\\{\}]
=== 4.4: BREUSCH-PAGAN TEST RESULTS ===
Test Statistic: 4.1871
P-value: 0.0407
Degrees of Freedom: 1

Conclusion: Reject H0 at α = 0.05. Evidence of heteroscedasticity.

Verification (statsmodels function): Stat = 65.1218, P-value = 0.0000


    \end{Verbatim}

    \begin{Verbatim}[commandchars=\\\{\}]
=== 4.5: RESIDUALS VS. FITTED VALUES ANALYSIS ===
Pattern interpretation:
- HOMOSCEDASTICITY: Points should be randomly scattered around the horizontal
line at y=0
- HETEROSCEDASTICITY indicators:
  * Funnel shape (variance increases or decreases with fitted values)
  * Curved patterns in the smoothing line
  * Clear clustering or systematic patterns

Variance in lowest third of fitted values: 17.2703
Variance in highest third of fitted values: 31.7984
Variance ratio (high/low): 1.8412
Interpretation: Ratio > 2 or < 0.5 suggests heteroscedasticity

    \end{Verbatim}

    \begin{center}
    \adjustimage{max size={0.9\linewidth}{0.9\paperheight}}{Math6450_Assignment1_copy_for_pdf_render_files/Math6450_Assignment1_copy_for_pdf_render_38_1.png}
    \end{center}
    { \hspace*{\fill} \\}
    
    \begin{Verbatim}[commandchars=\\\{\}]
=== 4.6: SCALE-LOCATION PLOT ANALYSIS ===
Evidence of changing variance:
- CONSTANT VARIANCE: Smoothing line should be roughly horizontal
- CHANGING VARIANCE indicators:
  * Upward or downward trend in smoothing line
  * Clear patterns or curves in the line

Correlation between fitted values and √|residuals|: 0.1507
Interpretation:
  * Moderate correlation suggests possible heteroscedasticity

    \end{Verbatim}

    \begin{center}
    \adjustimage{max size={0.9\linewidth}{0.9\paperheight}}{Math6450_Assignment1_copy_for_pdf_render_files/Math6450_Assignment1_copy_for_pdf_render_39_1.png}
    \end{center}
    { \hspace*{\fill} \\}
    
    \begin{Verbatim}[commandchars=\\\{\}]
=== COMPREHENSIVE HOMOSCEDASTICITY ASSESSMENT ===

TEST RESULTS SUMMARY:
1. Breusch-Pagan Test: Statistic = 4.1871, P-value = 0.0407
   → Reject H0 at α = 0.05. Evidence of heteroscedasticity.

2. Variance Ratio Analysis: 1.8412
   → Suggests homoscedasticity

3. Scale-Location Correlation: 0.1507
   → Moderate evidence of heteroscedasticity

RECOMMENDATIONS:
• Evidence suggests heteroscedasticity
• Consider transformations (log, Box-Cox)
• Use robust standard errors (White's correction)
• Consider weighted least squares regression
• Explore different model specifications

Note: Visual inspection of plots is crucial - statistical tests should be
combined with graphical analysis for complete assessment.
    \end{Verbatim}

    Task 4.3: Independence and Influence Diagnostics

    \begin{Verbatim}[commandchars=\\\{\}]
=== 4.7: DURBIN-WATSON TEST RESULTS ===
Durbin-Watson Statistic: 1.0784
First-order autocorrelation (ρ): 0.4608

INTERPRETATION:
→ Evidence of positive autocorrelation. Independence assumption may be violated.

Durbin-Watson Guidelines:
• DW ≈ 2.0: No autocorrelation (ideal)
• DW < 1.5: Strong positive autocorrelation
• DW > 2.5: Strong negative autocorrelation
• 1.5 ≤ DW ≤ 2.5: Acceptable range

    \end{Verbatim}

    \begin{Verbatim}[commandchars=\\\{\}]
=== 4.8: COOK'S DISTANCE ANALYSIS ===
Maximum Cook's Distance: 0.1657
Mean Cook's Distance: 0.0030
Standard Deviation: 0.0112

INFLUENTIAL OBSERVATIONS CRITERIA:
• Threshold 4/n = 4/506 = 0.0079
• Conservative threshold = 1.0

RESULTS:
• Observations with Cook's D > 4/n: 30 (5.9\%)
• Observations with Cook's D > 1.0: 0 (0.0\%)

CONCLUSION: Moderate Cook's distance values. Some observations may be
influential but not necessarily problematic.

TOP 5 MOST INFLUENTIAL OBSERVATIONS:
1. Observation 368: Cook's D = 0.1657
2. Observation 372: Cook's D = 0.0941
3. Observation 364: Cook's D = 0.0694
4. Observation 365: Cook's D = 0.0672
5. Observation 369: Cook's D = 0.0553

    \end{Verbatim}

    \begin{Verbatim}[commandchars=\\\{\}]
=== 4.9: HIGH LEVERAGE ANALYSIS ===
Number of parameters (p): 14
Sample size (n): 506
High leverage threshold (2p/n): 2 × 14 / 506 = 0.0553

HIGH LEVERAGE RESULTS:
• Observations with high leverage: 36
• Percentage of total sample: 7.1\%
• Maximum leverage value: 0.3060
• Mean leverage value: 0.0277

TOP 5 HIGHEST LEVERAGE OBSERVATIONS:
1. Observation 380: Leverage = 0.3060
2. Observation 418: Leverage = 0.1901
3. Observation 405: Leverage = 0.1564
4. Observation 410: Leverage = 0.1247
5. Observation 365: Leverage = 0.0985

    \end{Verbatim}

    4.10 Based on all assumption tests, is your linear regression model
valid for statistical inference? Summarize which assumptions are
satisfied and which (if any) are violated.

todo

    \begin{Verbatim}[commandchars=\\\{\}]
=== 4.10: COMPREHENSIVE MODEL VALIDATION SUMMARY ===

LINEAR REGRESSION ASSUMPTIONS ASSESSMENT:
==================================================
1. LINEARITY:
   □ Test method: Residuals vs. fitted plots, added variable plots
   □ Result: [Add your previous linearity test results]
   □ Status: [SATISFIED / VIOLATED / MARGINAL]

2. INDEPENDENCE OF RESIDUALS:
   ✓ Test method: Durbin-Watson test
   ✓ Result: DW = 1.0784
   ✓ Status: VIOLATED

3. HOMOSCEDASTICITY (Constant Variance):
   □ Test method: Breusch-Pagan test, residuals plots
   □ Result: [Add your previous homoscedasticity test results]
   □ Status: [SATISFIED / VIOLATED / MARGINAL]

4. NORMALITY OF RESIDUALS:
   □ Test method: Shapiro-Wilk, Q-Q plots, histograms
   □ Result: [Add your previous normality test results]
   □ Status: [SATISFIED / VIOLATED / MARGINAL]

5. NO MULTICOLLINEARITY:
   □ Test method: VIF analysis, correlation matrix
   □ Result: [Add your multicollinearity test results if available]
   □ Status: [SATISFIED / VIOLATED / MARGINAL]

6. NO EXCESSIVE INFLUENTIAL OBSERVATIONS:
   ✓ Test method: Cook's distance, leverage analysis
   ✓ Cook's D max: 0.1657
   ✓ High leverage obs: 36 (7.1\%)
   ✓ Status: MARGINAL - Some influential observations present

OVERALL MODEL VALIDITY FOR STATISTICAL INFERENCE:
==================================================
CURRENT ASSESSMENT (based on available tests):
• Assumptions checked: 2
• Assumptions satisfied: 0

RECOMMENDATIONS:
⚠ Some concerns with independence or influential observations

NEXT STEPS:
• Complete all assumption tests (linearity, homoscedasticity, normality)
• Consider remedial measures if assumptions are violated:
  - Data transformations (log, Box-Cox)
  - Robust regression methods
  - Remove or downweight influential observations
  - Use different modeling approaches if assumptions severely violated

Note: A complete assessment requires results from all assumption tests.
Update this summary once you have completed the full diagnostic suite.
    \end{Verbatim}

    \begin{center}
    \adjustimage{max size={0.9\linewidth}{0.9\paperheight}}{Math6450_Assignment1_copy_for_pdf_render_files/Math6450_Assignment1_copy_for_pdf_render_46_1.png}
    \end{center}
    { \hspace*{\fill} \\}
    
    Part 5: Predictions and Intervals

    \begin{Verbatim}[commandchars=\\\{\}]
============================================================
PREDICTIONS AND INTERVALS ANALYSIS
============================================================

DATASET OVERVIEW
------------------------------
Dataset shape: (506, 14)
Column names: ['crim', 'zn', 'indus', 'chas', 'nox', 'rm', 'age', 'dis', 'rad',
'tax', 'ptratio', 'b', 'lstat', 'medv']

Using 'medv' as target variable
Using 'lstat' as predictor variable (lstat)

=== SIMPLE LINEAR REGRESSION MODEL ===
Model: medv \textasciitilde{} lstat
R-squared: 0.5441
Regression equation: medv = 34.5538 + -0.9500 × lstat

    \end{Verbatim}

    \begin{Verbatim}[commandchars=\\\{\}]
=== 5.1: PREDICTION FOR LSTAT = 10\% ===
CALCULATION:
ŷ = β₀ + β₁ × X
ŷ = 34.5538 + -0.9500 × 10.0
ŷ = 25.0533

Predicted median home value for lstat = 10\%: \$25.05k

    \end{Verbatim}

    \begin{Verbatim}[commandchars=\\\{\}]
=== 5.2: 95\% CONFIDENCE INTERVAL FOR MEAN RESPONSE ===
CALCULATION DETAILS:
• Predicted value: 25.0533
• Standard error of mean: 0.2948
• t-critical (α=0.05, df=504.0): 1.9647
• Margin of error: 0.5792

95\% CONFIDENCE INTERVAL: [24.4741, 25.6326]
In dollars: [\$24.47k, \$25.63k]

INTERPRETATION:
We are 95\% confident that the mean median home value for all neighborhoods
with lstat = 10\% is between \$24.47k and \$25.63k.

    \end{Verbatim}

    \begin{Verbatim}[commandchars=\\\{\}]
=== 5.3: 95\% PREDICTION INTERVAL FOR INDIVIDUAL RESPONSE ===
CALCULATION DETAILS:
• Predicted value: 25.0533
• Standard error of prediction: 6.4803
• t-critical (α=0.05, df=504.0): 1.9647
• Margin of error: 12.7316

95\% PREDICTION INTERVAL: [12.3217, 37.7850]
In dollars: [\$12.32k, \$37.78k]

INTERVAL COMPARISON:
• Confidence interval width: 1.1584
• Prediction interval width: 25.4633
• Prediction interval is 21.98x wider than confidence interval

    \end{Verbatim}

    5.4 Explain the difference between a confidence interval and a
prediction interval in practical terms. When would you use each type?

todo

    \begin{Verbatim}[commandchars=\\\{\}]
=== 5.4: CONFIDENCE VS PREDICTION INTERVALS ===

CONCEPTUAL DIFFERENCES:
-------------------------

CONFIDENCE INTERVAL:
• Estimates uncertainty about the MEAN response for a given X value
• Answers: 'What is the average Y for all observations with this X?'
• Accounts for uncertainty in estimating the population mean
• Gets narrower as sample size increases
• Narrower interval (less uncertainty)

PREDICTION INTERVAL:
• Estimates uncertainty about an INDIVIDUAL response for a given X value
• Answers: 'What might Y be for a single new observation with this X?'
• Accounts for both estimation uncertainty AND individual variation
• Includes natural scatter around the regression line
• Wider interval (more uncertainty)

WHEN TO USE EACH:
--------------------

USE CONFIDENCE INTERVAL when:
• Estimating average outcomes for policy/planning
• Comparing mean responses between groups
• Making statements about population parameters
• Example: 'What's the average home value in 10\% lstat neighborhoods?'

USE PREDICTION INTERVAL when:
• Predicting outcomes for specific individuals/cases
• Setting bounds for individual forecasts
• Risk assessment for single observations
• Example: 'What might this specific house be worth?'

    \end{Verbatim}

    5.5 For lstat values of 5\%, 15\%, and 25\%, calculate point predictions
and comment on how the relationship changes across different levels of
the predictor variable

todo

    \begin{Verbatim}[commandchars=\\\{\}]
=== 5.5: PREDICTIONS AT MULTIPLE LSTAT VALUES ===

POINT PREDICTIONS:
--------------------
lstat = 5\%:
  → Predicted value: \$29.80k
  → 95\% CI: [\$29.01k, \$30.60k]
  → 95\% PI: [\$16.63k, \$42.98k]

lstat = 10\%:
  → Predicted value: \$25.05k
  → 95\% CI: [\$24.47k, \$25.63k]
  → 95\% PI: [\$12.32k, \$37.78k]

lstat = 15\%:
  → Predicted value: \$20.30k
  → 95\% CI: [\$19.73k, \$20.87k]
  → 95\% PI: [\$7.58k, \$33.02k]

lstat = 25\%:
  → Predicted value: \$10.80k
  → 95\% CI: [\$9.72k, \$11.89k]
  → 95\% PI: [\$-3.15k, \$24.75k]

RELATIONSHIP ANALYSIS:
-------------------------

Model slope (β₁): -0.9500
Interpretation: For each 1\% increase in lstat, median home value
decreases by \$0.95k on average

CHANGES BETWEEN LSTAT LEVELS:
• 5.0\% → 10.0\%: Change = \$-4.75k
  Rate: \$-0.95k per 1\% lstat increase
• 10.0\% → 15.0\%: Change = \$-4.75k
  Rate: \$-0.95k per 1\% lstat increase
• 15.0\% → 25.0\%: Change = \$-9.50k
  Rate: \$-0.95k per 1\% lstat increase

COMMENTS ON RELATIONSHIP:
• The relationship shows moderate negative association
• Linear relationship assumed constant across all lstat levels
• Higher lstat (more lower status population) associated with lower home values
    \end{Verbatim}

    \begin{center}
    \adjustimage{max size={0.9\linewidth}{0.9\paperheight}}{Math6450_Assignment1_copy_for_pdf_render_files/Math6450_Assignment1_copy_for_pdf_render_56_0.png}
    \end{center}
    { \hspace*{\fill} \\}
    
    \begin{Verbatim}[commandchars=\\\{\}]
=== PREDICTIONS SUMMARY TABLE ===

DETAILED PREDICTIONS TABLE:
 lstat  prediction  ci\_lower  ci\_upper  pi\_lower  pi\_upper  ci\_width  pi\_width
width\_ratio
     5      29.804    29.007    30.600    16.627    42.980     1.592    26.353
16.550
    10      25.053    24.474    25.633    12.322    37.785     1.158    25.463
21.981
    15      20.303    19.732    20.875     7.585    33.021     1.143    25.436
22.254
    25      10.803     9.717    11.888    -3.148    24.754     2.170    27.902
12.856

KEY INSIGHTS:
• As lstat increases, predicted home values decrease
• Prediction intervals are consistently 18.4x wider than confidence intervals
• The linear relationship appears moderate (R² = 0.544)

=== MODEL ASSUMPTIONS REMINDER ===
For these intervals to be valid, ensure:
• Linear relationship between variables
• Independence of residuals
• Homoscedasticity (constant variance)
• Normality of residuals
• No influential outliers
    \end{Verbatim}


    % Add a bibliography block to the postdoc
    
    
    
\end{document}
