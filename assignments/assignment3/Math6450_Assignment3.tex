\documentclass[11pt, twocolumn]{article}

    \usepackage[breakable]{tcolorbox}
    \usepackage{parskip} % Stop auto-indenting (to mimic markdown behaviour)
    

    % Basic figure setup, for now with no caption control since it's done
    % automatically by Pandoc (which extracts ![](path) syntax from Markdown).
    \usepackage{graphicx}
    % Keep aspect ratio if custom image width or height is specified
    \setkeys{Gin}{keepaspectratio}
    % Maintain compatibility with old templates. Remove in nbconvert 6.0
    \let\Oldincludegraphics\includegraphics
    % Ensure that by default, figures have no caption (until we provide a
    % proper Figure object with a Caption API and a way to capture that
    % in the conversion process - todo).
    \usepackage{caption}
    \DeclareCaptionFormat{nocaption}{}
    \captionsetup{format=nocaption,aboveskip=0pt,belowskip=0pt}

    \usepackage{float}
    \floatplacement{figure}{H} % forces figures to be placed at the correct location
    \usepackage{xcolor} % Allow colors to be defined
    \usepackage{enumerate} % Needed for markdown enumerations to work
    \usepackage{geometry} % Used to adjust the document margins
    \usepackage{amsmath} % Equations
    \usepackage{amssymb} % Equations
    \usepackage{textcomp} % defines textquotesingle
    % Hack from http://tex.stackexchange.com/a/47451/13684:
    \AtBeginDocument{%
        \def\PYZsq{\textquotesingle}% Upright quotes in Pygmentized code
    }
    \usepackage{upquote} % Upright quotes for verbatim code
    \usepackage{eurosym} % defines \euro

    \usepackage{iftex}
    \ifPDFTeX
        \usepackage[T1]{fontenc}
        \IfFileExists{alphabeta.sty}{
              \usepackage{alphabeta}
          }{
              \usepackage[mathletters]{ucs}
              \usepackage[utf8x]{inputenc}
          }
    \else
        \usepackage{fontspec}
        \usepackage{unicode-math}
    \fi

    \usepackage{fancyvrb} % verbatim replacement that allows latex
    \usepackage{grffile} % extends the file name processing of package graphics
                         % to support a larger range
    \makeatletter % fix for old versions of grffile with XeLaTeX
    \@ifpackagelater{grffile}{2019/11/01}
    {
      % Do nothing on new versions
    }
    {
      \def\Gread@@xetex#1{%
        \IfFileExists{"\Gin@base".bb}%
        {\Gread@eps{\Gin@base.bb}}%
        {\Gread@@xetex@aux#1}%
      }
    }
    \makeatother
    \usepackage[Export]{adjustbox} % Used to constrain images to a maximum size
    \adjustboxset{max size={0.9\linewidth}{0.9\paperheight}}

    % The hyperref package gives us a pdf with properly built
    % internal navigation ('pdf bookmarks' for the table of contents,
    % internal cross-reference links, web links for URLs, etc.)
    \usepackage{hyperref}
    % The default LaTeX title has an obnoxious amount of whitespace. By default,
    % titling removes some of it. It also provides customization options.
    \usepackage{titling}
    \usepackage{longtable} % longtable support required by pandoc >1.10
    \usepackage{booktabs}  % table support for pandoc > 1.12.2
    \usepackage{array}     % table support for pandoc >= 2.11.3
    \usepackage{calc}      % table minipage width calculation for pandoc >= 2.11.1
    \usepackage[inline]{enumitem} % IRkernel/repr support (it uses the enumerate* environment)
    \usepackage[normalem]{ulem} % ulem is needed to support strikethroughs (\sout)
                                % normalem makes italics be italics, not underlines
    \usepackage{soul}      % strikethrough (\st) support for pandoc >= 3.0.0
    \usepackage{mathrsfs}
    

    
    % Colors for the hyperref package
    \definecolor{urlcolor}{rgb}{0,.145,.698}
    \definecolor{linkcolor}{rgb}{.71,0.21,0.01}
    \definecolor{citecolor}{rgb}{.12,.54,.11}

    % ANSI colors
    \definecolor{ansi-black}{HTML}{3E424D}
    \definecolor{ansi-black-intense}{HTML}{282C36}
    \definecolor{ansi-red}{HTML}{E75C58}
    \definecolor{ansi-red-intense}{HTML}{B22B31}
    \definecolor{ansi-green}{HTML}{00A250}
    \definecolor{ansi-green-intense}{HTML}{007427}
    \definecolor{ansi-yellow}{HTML}{DDB62B}
    \definecolor{ansi-yellow-intense}{HTML}{B27D12}
    \definecolor{ansi-blue}{HTML}{208FFB}
    \definecolor{ansi-blue-intense}{HTML}{0065CA}
    \definecolor{ansi-magenta}{HTML}{D160C4}
    \definecolor{ansi-magenta-intense}{HTML}{A03196}
    \definecolor{ansi-cyan}{HTML}{60C6C8}
    \definecolor{ansi-cyan-intense}{HTML}{258F8F}
    \definecolor{ansi-white}{HTML}{C5C1B4}
    \definecolor{ansi-white-intense}{HTML}{A1A6B2}
    \definecolor{ansi-default-inverse-fg}{HTML}{FFFFFF}
    \definecolor{ansi-default-inverse-bg}{HTML}{000000}

    % common color for the border for error outputs.
    \definecolor{outerrorbackground}{HTML}{FFDFDF}

    % commands and environments needed by pandoc snippets
    % extracted from the output of `pandoc -s`
    \providecommand{\tightlist}{%
      \setlength{\itemsep}{0pt}\setlength{\parskip}{0pt}}
    \DefineVerbatimEnvironment{Highlighting}{Verbatim}{commandchars=\\\{\}}
    % Add ',fontsize=\small' for more characters per line
    \newenvironment{Shaded}{}{}
    \newcommand{\KeywordTok}[1]{\textcolor[rgb]{0.00,0.44,0.13}{\textbf{{#1}}}}
    \newcommand{\DataTypeTok}[1]{\textcolor[rgb]{0.56,0.13,0.00}{{#1}}}
    \newcommand{\DecValTok}[1]{\textcolor[rgb]{0.25,0.63,0.44}{{#1}}}
    \newcommand{\BaseNTok}[1]{\textcolor[rgb]{0.25,0.63,0.44}{{#1}}}
    \newcommand{\FloatTok}[1]{\textcolor[rgb]{0.25,0.63,0.44}{{#1}}}
    \newcommand{\CharTok}[1]{\textcolor[rgb]{0.25,0.44,0.63}{{#1}}}
    \newcommand{\StringTok}[1]{\textcolor[rgb]{0.25,0.44,0.63}{{#1}}}
    \newcommand{\CommentTok}[1]{\textcolor[rgb]{0.38,0.63,0.69}{\textit{{#1}}}}
    \newcommand{\OtherTok}[1]{\textcolor[rgb]{0.00,0.44,0.13}{{#1}}}
    \newcommand{\AlertTok}[1]{\textcolor[rgb]{1.00,0.00,0.00}{\textbf{{#1}}}}
    \newcommand{\FunctionTok}[1]{\textcolor[rgb]{0.02,0.16,0.49}{{#1}}}
    \newcommand{\RegionMarkerTok}[1]{{#1}}
    \newcommand{\ErrorTok}[1]{\textcolor[rgb]{1.00,0.00,0.00}{\textbf{{#1}}}}
    \newcommand{\NormalTok}[1]{{#1}}

    % Additional commands for more recent versions of Pandoc
    \newcommand{\ConstantTok}[1]{\textcolor[rgb]{0.53,0.00,0.00}{{#1}}}
    \newcommand{\SpecialCharTok}[1]{\textcolor[rgb]{0.25,0.44,0.63}{{#1}}}
    \newcommand{\VerbatimStringTok}[1]{\textcolor[rgb]{0.25,0.44,0.63}{{#1}}}
    \newcommand{\SpecialStringTok}[1]{\textcolor[rgb]{0.73,0.40,0.53}{{#1}}}
    \newcommand{\ImportTok}[1]{{#1}}
    \newcommand{\DocumentationTok}[1]{\textcolor[rgb]{0.73,0.13,0.13}{\textit{{#1}}}}
    \newcommand{\AnnotationTok}[1]{\textcolor[rgb]{0.38,0.63,0.69}{\textbf{\textit{{#1}}}}}
    \newcommand{\CommentVarTok}[1]{\textcolor[rgb]{0.38,0.63,0.69}{\textbf{\textit{{#1}}}}}
    \newcommand{\VariableTok}[1]{\textcolor[rgb]{0.10,0.09,0.49}{{#1}}}
    \newcommand{\ControlFlowTok}[1]{\textcolor[rgb]{0.00,0.44,0.13}{\textbf{{#1}}}}
    \newcommand{\OperatorTok}[1]{\textcolor[rgb]{0.40,0.40,0.40}{{#1}}}
    \newcommand{\BuiltInTok}[1]{{#1}}
    \newcommand{\ExtensionTok}[1]{{#1}}
    \newcommand{\PreprocessorTok}[1]{\textcolor[rgb]{0.74,0.48,0.00}{{#1}}}
    \newcommand{\AttributeTok}[1]{\textcolor[rgb]{0.49,0.56,0.16}{{#1}}}
    \newcommand{\InformationTok}[1]{\textcolor[rgb]{0.38,0.63,0.69}{\textbf{\textit{{#1}}}}}
    \newcommand{\WarningTok}[1]{\textcolor[rgb]{0.38,0.63,0.69}{\textbf{\textit{{#1}}}}}
    \makeatletter
    \newsavebox\pandoc@box
    \newcommand*\pandocbounded[1]{%
      \sbox\pandoc@box{#1}%
      % scaling factors for width and height
      \Gscale@div\@tempa\textheight{\dimexpr\ht\pandoc@box+\dp\pandoc@box\relax}%
      \Gscale@div\@tempb\linewidth{\wd\pandoc@box}%
      % select the smaller of both
      \ifdim\@tempb\p@<\@tempa\p@
        \let\@tempa\@tempb
      \fi
      % scaling accordingly (\@tempa < 1)
      \ifdim\@tempa\p@<\p@
        \scalebox{\@tempa}{\usebox\pandoc@box}%
      % scaling not needed, use as it is
      \else
        \usebox{\pandoc@box}%
      \fi
    }
    \makeatother

    % Define a nice break command that doesn't care if a line doesn't already
    % exist.
    \def\br{\hspace*{\fill} \\* }
    % Math Jax compatibility definitions
    \def\gt{>}
    \def\lt{<}
    \let\Oldtex\TeX
    \let\Oldlatex\LaTeX
    \renewcommand{\TeX}{\textrm{\Oldtex}}
    \renewcommand{\LaTeX}{\textrm{\Oldlatex}}
    % Document parameters
    % Document title
    \title{Math6450\_Assignment3}
    
    
    
    
    
    
    
% Pygments definitions
\makeatletter
\def\PY@reset{\let\PY@it=\relax \let\PY@bf=\relax%
    \let\PY@ul=\relax \let\PY@tc=\relax%
    \let\PY@bc=\relax \let\PY@ff=\relax}
\def\PY@tok#1{\csname PY@tok@#1\endcsname}
\def\PY@toks#1+{\ifx\relax#1\empty\else%
    \PY@tok{#1}\expandafter\PY@toks\fi}
\def\PY@do#1{\PY@bc{\PY@tc{\PY@ul{%
    \PY@it{\PY@bf{\PY@ff{#1}}}}}}}
\def\PY#1#2{\PY@reset\PY@toks#1+\relax+\PY@do{#2}}

\@namedef{PY@tok@w}{\def\PY@tc##1{\textcolor[rgb]{0.73,0.73,0.73}{##1}}}
\@namedef{PY@tok@c}{\let\PY@it=\textit\def\PY@tc##1{\textcolor[rgb]{0.24,0.48,0.48}{##1}}}
\@namedef{PY@tok@cp}{\def\PY@tc##1{\textcolor[rgb]{0.61,0.40,0.00}{##1}}}
\@namedef{PY@tok@k}{\let\PY@bf=\textbf\def\PY@tc##1{\textcolor[rgb]{0.00,0.50,0.00}{##1}}}
\@namedef{PY@tok@kp}{\def\PY@tc##1{\textcolor[rgb]{0.00,0.50,0.00}{##1}}}
\@namedef{PY@tok@kt}{\def\PY@tc##1{\textcolor[rgb]{0.69,0.00,0.25}{##1}}}
\@namedef{PY@tok@o}{\def\PY@tc##1{\textcolor[rgb]{0.40,0.40,0.40}{##1}}}
\@namedef{PY@tok@ow}{\let\PY@bf=\textbf\def\PY@tc##1{\textcolor[rgb]{0.67,0.13,1.00}{##1}}}
\@namedef{PY@tok@nb}{\def\PY@tc##1{\textcolor[rgb]{0.00,0.50,0.00}{##1}}}
\@namedef{PY@tok@nf}{\def\PY@tc##1{\textcolor[rgb]{0.00,0.00,1.00}{##1}}}
\@namedef{PY@tok@nc}{\let\PY@bf=\textbf\def\PY@tc##1{\textcolor[rgb]{0.00,0.00,1.00}{##1}}}
\@namedef{PY@tok@nn}{\let\PY@bf=\textbf\def\PY@tc##1{\textcolor[rgb]{0.00,0.00,1.00}{##1}}}
\@namedef{PY@tok@ne}{\let\PY@bf=\textbf\def\PY@tc##1{\textcolor[rgb]{0.80,0.25,0.22}{##1}}}
\@namedef{PY@tok@nv}{\def\PY@tc##1{\textcolor[rgb]{0.10,0.09,0.49}{##1}}}
\@namedef{PY@tok@no}{\def\PY@tc##1{\textcolor[rgb]{0.53,0.00,0.00}{##1}}}
\@namedef{PY@tok@nl}{\def\PY@tc##1{\textcolor[rgb]{0.46,0.46,0.00}{##1}}}
\@namedef{PY@tok@ni}{\let\PY@bf=\textbf\def\PY@tc##1{\textcolor[rgb]{0.44,0.44,0.44}{##1}}}
\@namedef{PY@tok@na}{\def\PY@tc##1{\textcolor[rgb]{0.41,0.47,0.13}{##1}}}
\@namedef{PY@tok@nt}{\let\PY@bf=\textbf\def\PY@tc##1{\textcolor[rgb]{0.00,0.50,0.00}{##1}}}
\@namedef{PY@tok@nd}{\def\PY@tc##1{\textcolor[rgb]{0.67,0.13,1.00}{##1}}}
\@namedef{PY@tok@s}{\def\PY@tc##1{\textcolor[rgb]{0.73,0.13,0.13}{##1}}}
\@namedef{PY@tok@sd}{\let\PY@it=\textit\def\PY@tc##1{\textcolor[rgb]{0.73,0.13,0.13}{##1}}}
\@namedef{PY@tok@si}{\let\PY@bf=\textbf\def\PY@tc##1{\textcolor[rgb]{0.64,0.35,0.47}{##1}}}
\@namedef{PY@tok@se}{\let\PY@bf=\textbf\def\PY@tc##1{\textcolor[rgb]{0.67,0.36,0.12}{##1}}}
\@namedef{PY@tok@sr}{\def\PY@tc##1{\textcolor[rgb]{0.64,0.35,0.47}{##1}}}
\@namedef{PY@tok@ss}{\def\PY@tc##1{\textcolor[rgb]{0.10,0.09,0.49}{##1}}}
\@namedef{PY@tok@sx}{\def\PY@tc##1{\textcolor[rgb]{0.00,0.50,0.00}{##1}}}
\@namedef{PY@tok@m}{\def\PY@tc##1{\textcolor[rgb]{0.40,0.40,0.40}{##1}}}
\@namedef{PY@tok@gh}{\let\PY@bf=\textbf\def\PY@tc##1{\textcolor[rgb]{0.00,0.00,0.50}{##1}}}
\@namedef{PY@tok@gu}{\let\PY@bf=\textbf\def\PY@tc##1{\textcolor[rgb]{0.50,0.00,0.50}{##1}}}
\@namedef{PY@tok@gd}{\def\PY@tc##1{\textcolor[rgb]{0.63,0.00,0.00}{##1}}}
\@namedef{PY@tok@gi}{\def\PY@tc##1{\textcolor[rgb]{0.00,0.52,0.00}{##1}}}
\@namedef{PY@tok@gr}{\def\PY@tc##1{\textcolor[rgb]{0.89,0.00,0.00}{##1}}}
\@namedef{PY@tok@ge}{\let\PY@it=\textit}
\@namedef{PY@tok@gs}{\let\PY@bf=\textbf}
\@namedef{PY@tok@ges}{\let\PY@bf=\textbf\let\PY@it=\textit}
\@namedef{PY@tok@gp}{\let\PY@bf=\textbf\def\PY@tc##1{\textcolor[rgb]{0.00,0.00,0.50}{##1}}}
\@namedef{PY@tok@go}{\def\PY@tc##1{\textcolor[rgb]{0.44,0.44,0.44}{##1}}}
\@namedef{PY@tok@gt}{\def\PY@tc##1{\textcolor[rgb]{0.00,0.27,0.87}{##1}}}
\@namedef{PY@tok@err}{\def\PY@bc##1{{\setlength{\fboxsep}{\string -\fboxrule}\fcolorbox[rgb]{1.00,0.00,0.00}{1,1,1}{\strut ##1}}}}
\@namedef{PY@tok@kc}{\let\PY@bf=\textbf\def\PY@tc##1{\textcolor[rgb]{0.00,0.50,0.00}{##1}}}
\@namedef{PY@tok@kd}{\let\PY@bf=\textbf\def\PY@tc##1{\textcolor[rgb]{0.00,0.50,0.00}{##1}}}
\@namedef{PY@tok@kn}{\let\PY@bf=\textbf\def\PY@tc##1{\textcolor[rgb]{0.00,0.50,0.00}{##1}}}
\@namedef{PY@tok@kr}{\let\PY@bf=\textbf\def\PY@tc##1{\textcolor[rgb]{0.00,0.50,0.00}{##1}}}
\@namedef{PY@tok@bp}{\def\PY@tc##1{\textcolor[rgb]{0.00,0.50,0.00}{##1}}}
\@namedef{PY@tok@fm}{\def\PY@tc##1{\textcolor[rgb]{0.00,0.00,1.00}{##1}}}
\@namedef{PY@tok@vc}{\def\PY@tc##1{\textcolor[rgb]{0.10,0.09,0.49}{##1}}}
\@namedef{PY@tok@vg}{\def\PY@tc##1{\textcolor[rgb]{0.10,0.09,0.49}{##1}}}
\@namedef{PY@tok@vi}{\def\PY@tc##1{\textcolor[rgb]{0.10,0.09,0.49}{##1}}}
\@namedef{PY@tok@vm}{\def\PY@tc##1{\textcolor[rgb]{0.10,0.09,0.49}{##1}}}
\@namedef{PY@tok@sa}{\def\PY@tc##1{\textcolor[rgb]{0.73,0.13,0.13}{##1}}}
\@namedef{PY@tok@sb}{\def\PY@tc##1{\textcolor[rgb]{0.73,0.13,0.13}{##1}}}
\@namedef{PY@tok@sc}{\def\PY@tc##1{\textcolor[rgb]{0.73,0.13,0.13}{##1}}}
\@namedef{PY@tok@dl}{\def\PY@tc##1{\textcolor[rgb]{0.73,0.13,0.13}{##1}}}
\@namedef{PY@tok@s2}{\def\PY@tc##1{\textcolor[rgb]{0.73,0.13,0.13}{##1}}}
\@namedef{PY@tok@sh}{\def\PY@tc##1{\textcolor[rgb]{0.73,0.13,0.13}{##1}}}
\@namedef{PY@tok@s1}{\def\PY@tc##1{\textcolor[rgb]{0.73,0.13,0.13}{##1}}}
\@namedef{PY@tok@mb}{\def\PY@tc##1{\textcolor[rgb]{0.40,0.40,0.40}{##1}}}
\@namedef{PY@tok@mf}{\def\PY@tc##1{\textcolor[rgb]{0.40,0.40,0.40}{##1}}}
\@namedef{PY@tok@mh}{\def\PY@tc##1{\textcolor[rgb]{0.40,0.40,0.40}{##1}}}
\@namedef{PY@tok@mi}{\def\PY@tc##1{\textcolor[rgb]{0.40,0.40,0.40}{##1}}}
\@namedef{PY@tok@il}{\def\PY@tc##1{\textcolor[rgb]{0.40,0.40,0.40}{##1}}}
\@namedef{PY@tok@mo}{\def\PY@tc##1{\textcolor[rgb]{0.40,0.40,0.40}{##1}}}
\@namedef{PY@tok@ch}{\let\PY@it=\textit\def\PY@tc##1{\textcolor[rgb]{0.24,0.48,0.48}{##1}}}
\@namedef{PY@tok@cm}{\let\PY@it=\textit\def\PY@tc##1{\textcolor[rgb]{0.24,0.48,0.48}{##1}}}
\@namedef{PY@tok@cpf}{\let\PY@it=\textit\def\PY@tc##1{\textcolor[rgb]{0.24,0.48,0.48}{##1}}}
\@namedef{PY@tok@c1}{\let\PY@it=\textit\def\PY@tc##1{\textcolor[rgb]{0.24,0.48,0.48}{##1}}}
\@namedef{PY@tok@cs}{\let\PY@it=\textit\def\PY@tc##1{\textcolor[rgb]{0.24,0.48,0.48}{##1}}}

\def\PYZbs{\char`\\}
\def\PYZus{\char`\_}
\def\PYZob{\char`\{}
\def\PYZcb{\char`\}}
\def\PYZca{\char`\^}
\def\PYZam{\char`\&}
\def\PYZlt{\char`\<}
\def\PYZgt{\char`\>}
\def\PYZsh{\char`\#}
\def\PYZpc{\char`\%}
\def\PYZdl{\char`\$}
\def\PYZhy{\char`\-}
\def\PYZsq{\char`\'}
\def\PYZdq{\char`\"}
\def\PYZti{\char`\~}
% for compatibility with earlier versions
\def\PYZat{@}
\def\PYZlb{[}
\def\PYZrb{]}
\makeatother


    % For linebreaks inside Verbatim environment from package fancyvrb.
    \makeatletter
        \newbox\Wrappedcontinuationbox
        \newbox\Wrappedvisiblespacebox
        \newcommand*\Wrappedvisiblespace {\textcolor{red}{\textvisiblespace}}
        \newcommand*\Wrappedcontinuationsymbol {\textcolor{red}{\llap{\tiny$\m@th\hookrightarrow$}}}
        \newcommand*\Wrappedcontinuationindent {3ex }
        \newcommand*\Wrappedafterbreak {\kern\Wrappedcontinuationindent\copy\Wrappedcontinuationbox}
        % Take advantage of the already applied Pygments mark-up to insert
        % potential linebreaks for TeX processing.
        %        {, <, #, %, $, ' and ": go to next line.
        %        _, }, ^, &, >, - and ~: stay at end of broken line.
        % Use of \textquotesingle for straight quote.
        \newcommand*\Wrappedbreaksatspecials {%
            \def\PYGZus{\discretionary{\char`\_}{\Wrappedafterbreak}{\char`\_}}%
            \def\PYGZob{\discretionary{}{\Wrappedafterbreak\char`\{}{\char`\{}}%
            \def\PYGZcb{\discretionary{\char`\}}{\Wrappedafterbreak}{\char`\}}}%
            \def\PYGZca{\discretionary{\char`\^}{\Wrappedafterbreak}{\char`\^}}%
            \def\PYGZam{\discretionary{\char`\&}{\Wrappedafterbreak}{\char`\&}}%
            \def\PYGZlt{\discretionary{}{\Wrappedafterbreak\char`\<}{\char`\<}}%
            \def\PYGZgt{\discretionary{\char`\>}{\Wrappedafterbreak}{\char`\>}}%
            \def\PYGZsh{\discretionary{}{\Wrappedafterbreak\char`\#}{\char`\#}}%
            \def\PYGZpc{\discretionary{}{\Wrappedafterbreak\char`\%}{\char`\%}}%
            \def\PYGZdl{\discretionary{}{\Wrappedafterbreak\char`\$}{\char`\$}}%
            \def\PYGZhy{\discretionary{\char`\-}{\Wrappedafterbreak}{\char`\-}}%
            \def\PYGZsq{\discretionary{}{\Wrappedafterbreak\textquotesingle}{\textquotesingle}}%
            \def\PYGZdq{\discretionary{}{\Wrappedafterbreak\char`\"}{\char`\"}}%
            \def\PYGZti{\discretionary{\char`\~}{\Wrappedafterbreak}{\char`\~}}%
        }
        % Some characters . , ; ? ! / are not pygmentized.
        % This macro makes them "active" and they will insert potential linebreaks
        \newcommand*\Wrappedbreaksatpunct {%
            \lccode`\~`\.\lowercase{\def~}{\discretionary{\hbox{\char`\.}}{\Wrappedafterbreak}{\hbox{\char`\.}}}%
            \lccode`\~`\,\lowercase{\def~}{\discretionary{\hbox{\char`\,}}{\Wrappedafterbreak}{\hbox{\char`\,}}}%
            \lccode`\~`\;\lowercase{\def~}{\discretionary{\hbox{\char`\;}}{\Wrappedafterbreak}{\hbox{\char`\;}}}%
            \lccode`\~`\:\lowercase{\def~}{\discretionary{\hbox{\char`\:}}{\Wrappedafterbreak}{\hbox{\char`\:}}}%
            \lccode`\~`\?\lowercase{\def~}{\discretionary{\hbox{\char`\?}}{\Wrappedafterbreak}{\hbox{\char`\?}}}%
            \lccode`\~`\!\lowercase{\def~}{\discretionary{\hbox{\char`\!}}{\Wrappedafterbreak}{\hbox{\char`\!}}}%
            \lccode`\~`\/\lowercase{\def~}{\discretionary{\hbox{\char`\/}}{\Wrappedafterbreak}{\hbox{\char`\/}}}%
            \catcode`\.\active
            \catcode`\,\active
            \catcode`\;\active
            \catcode`\:\active
            \catcode`\?\active
            \catcode`\!\active
            \catcode`\/\active
            \lccode`\~`\~
        }
    \makeatother

    \let\OriginalVerbatim=\Verbatim
    \makeatletter
    \renewcommand{\Verbatim}[1][1]{%
        %\parskip\z@skip
        \sbox\Wrappedcontinuationbox {\Wrappedcontinuationsymbol}%
        \sbox\Wrappedvisiblespacebox {\FV@SetupFont\Wrappedvisiblespace}%
        \def\FancyVerbFormatLine ##1{\hsize\linewidth
            \vtop{\raggedright\hyphenpenalty\z@\exhyphenpenalty\z@
                \doublehyphendemerits\z@\finalhyphendemerits\z@
                \strut ##1\strut}%
        }%
        % If the linebreak is at a space, the latter will be displayed as visible
        % space at end of first line, and a continuation symbol starts next line.
        % Stretch/shrink are however usually zero for typewriter font.
        \def\FV@Space {%
            \nobreak\hskip\z@ plus\fontdimen3\font minus\fontdimen4\font
            \discretionary{\copy\Wrappedvisiblespacebox}{\Wrappedafterbreak}
            {\kern\fontdimen2\font}%
        }%

        % Allow breaks at special characters using \PYG... macros.
        \Wrappedbreaksatspecials
        % Breaks at punctuation characters . , ; ? ! and / need catcode=\active
        \OriginalVerbatim[#1,codes*=\Wrappedbreaksatpunct]%
    }
    \makeatother

    % Exact colors from NB
    \definecolor{incolor}{HTML}{303F9F}
    \definecolor{outcolor}{HTML}{D84315}
    \definecolor{cellborder}{HTML}{CFCFCF}
    \definecolor{cellbackground}{HTML}{F7F7F7}

    % prompt
    \makeatletter
    \newcommand{\boxspacing}{\kern\kvtcb@left@rule\kern\kvtcb@boxsep}
    \makeatother
    \newcommand{\prompt}[4]{
        {\ttfamily\llap{{\color{#2}[#3]:\hspace{3pt}#4}}\vspace{-\baselineskip}}
    }
    

    
    % Prevent overflowing lines due to hard-to-break entities
    \sloppy
    % Setup hyperref package
    \hypersetup{
      breaklinks=true,  % so long urls are correctly broken across lines
      colorlinks=true,
      urlcolor=urlcolor,
      linkcolor=linkcolor,
      citecolor=citecolor,
      }
    % Slightly bigger margins than the latex defaults
    
    \geometry{verbose,tmargin=1in,bmargin=1in,lmargin=1in,rmargin=1in}
    
    

\begin{document}
    
    \maketitle
    
    

    
    \subparagraph{Exploratory Data Analysis
(EDA):}\label{exploratory-data-analysis-eda}

    The data was loaded in and filtered to remove all columns except for
State Name, and county Name. Then additional steps were taken to only
grab rows with Utah, and Weber in their state and county name. 2022 had
365 rows, 2023 had 365 rows, and 2024 had 366 rows. This meant that
there was a valid data entry for every single day, and this was proven
to be true since there were no missing or NA values in these datasets.

    \begin{center}
    \adjustimage{max size={0.9\linewidth}{0.9\paperheight}}{Math6450_Assignment3_files/Math6450_Assignment3_6_0.png}
    \end{center}
    { \hspace*{\fill} \\}
    
    The daily AQI shows that there is quite a bit of fluctuations day to
day, and that each year follows a similar pattern. This means there
seasonality is present in this data.

    \begin{center}
    \adjustimage{max size={0.9\linewidth}{0.9\paperheight}}{Math6450_Assignment3_files/Math6450_Assignment3_8_0.png}
    \end{center}
    { \hspace*{\fill} \\}
    
    Using the monthly average of the AQI shows a smoother line and a bigger
picture view to how the data has changed over the years. However, it can
smooth the line too much and it may not be able to capture short-term
AQI fluctuations and changes. Smoothing out the graph also shows the
very obvious point where we have a seasonal trend. The month July
consistently jumps up in AQI.

    \begin{center}
    \adjustimage{max size={0.9\linewidth}{0.9\paperheight}}{Math6450_Assignment3_files/Math6450_Assignment3_10_0.png}
    \end{center}
    { \hspace*{\fill} \\}
    
    Looking at the weekly changes in AQI shows that there seems to be some
seasonality and trends in the data, but we can see short term
fluctuations as well. This may be an applicable middle ground since the
lines aren't as smooth as they are in the monthly AQI plot.

    \subparagraph{Reason for Aggregation (if
applicable):}\label{reason-for-aggregation-if-applicable}

    Upon initial data exploration, while there is enough data for daily time
series modeling, the distribution was skewed right, and the daily trends
were quite jagged. So, when graphing both monthly, and weekly averages
the monthly data looked a little to smooth, and weekly data looked to be
the most promising since there are still obvious seasonly patterns and
the data didn't look so smoothed over it wouldn't capture other trends.

So, in an attempt to capture as much AQI patterns as possible this the
weekly and monthly data will be used to compare and contrast results.

This means that from the daily AQI data we will aggregate the weekly and
monthly means for analyzing, and AQI forcasting.

    \begin{center}
    \adjustimage{max size={0.9\linewidth}{0.9\paperheight}}{Math6450_Assignment3_files/Math6450_Assignment3_14_0.png}
    \end{center}
    { \hspace*{\fill} \\}
    
    The daily AQI variance is a little skewed to the right, and so
aggregation will be a good idea for making the variance more normal.

    \begin{center}
    \adjustimage{max size={0.9\linewidth}{0.9\paperheight}}{Math6450_Assignment3_files/Math6450_Assignment3_16_0.png}
    \end{center}
    { \hspace*{\fill} \\}
    
    The monthly AQI data shows a similar right-skewed pattern, but the tail
is a little smaller now.

    \begin{center}
    \adjustimage{max size={0.9\linewidth}{0.9\paperheight}}{Math6450_Assignment3_files/Math6450_Assignment3_18_0.png}
    \end{center}
    { \hspace*{\fill} \\}
    
    The variance is not as prevalent, but still there, so we will move to
log transformations to ensure the data is stationary.

    \subparagraph{Transformations:}\label{transformations}

    \begin{center}
    \adjustimage{max size={0.9\linewidth}{0.9\paperheight}}{Math6450_Assignment3_files/Math6450_Assignment3_21_0.png}
    \end{center}
    { \hspace*{\fill} \\}
    
    The distributuion looks quite normal now after log transformation.

    \begin{center}
    \adjustimage{max size={0.9\linewidth}{0.9\paperheight}}{Math6450_Assignment3_files/Math6450_Assignment3_23_0.png}
    \end{center}
    { \hspace*{\fill} \\}
    
    The log transformed weekly data also shows a more normal variance when
plotted.

    \subparagraph{Seasonal Patterns:}\label{seasonal-patterns}

    Well it passed after log transforming so we will still check for
seasonality.

    We first noticed of the seasonality of the data in the Daily AQI graph,
and it was made clear there was seasonality when graphing the aggregated
monthly avg AQI dataset. In order to get rid of this seasonality seasonl
differencing will be needed to ensure this datat set is staionary and
ready for modeling.

    \begin{center}
    \adjustimage{max size={0.9\linewidth}{0.9\paperheight}}{Math6450_Assignment3_files/Math6450_Assignment3_28_0.png}
    \end{center}
    { \hspace*{\fill} \\}
    
    We can interpret the adf statistic as the more negative, the more
evidence there is that the data is stationary. A smaller p-value means
that there is high evidence we can reject the null resulting in a
stationary time series. Both of these numbers are low so we can assume
we have stationary time series data.

    \begin{center}
    \adjustimage{max size={0.9\linewidth}{0.9\paperheight}}{Math6450_Assignment3_files/Math6450_Assignment3_31_0.png}
    \end{center}
    { \hspace*{\fill} \\}
    
    The seasonal decomposition shows there are several spikes in data here
and there, but most importantly there is oscillating behavior in the
visuals which means air quality does suffer from seasonality.

    \begin{center}
    \adjustimage{max size={0.9\linewidth}{0.9\paperheight}}{Math6450_Assignment3_files/Math6450_Assignment3_33_0.png}
    \end{center}
    { \hspace*{\fill} \\}
    
    Since the trend and seasonality of this data was so obvious when looking
at the values graphed, I wanted to also see what the monthly data looked
like, and it is much more obvious that there is a major trend going on
in this data.

    \begin{center}
    \adjustimage{max size={0.9\linewidth}{0.9\paperheight}}{Math6450_Assignment3_files/Math6450_Assignment3_35_0.png}
    \end{center}
    { \hspace*{\fill} \\}
    
    We knew that there was seasonality in the data so by taking the season
first difference we have adf stats that are more negative, and p-values
that are even smaller, showing that the data has significant evidence
that it is stationary.

    Comparing the ADF p-value for the weekly aqi/logged-aqi values and the
monthly aqi/logged-aqi values show that the p-value is very small
meaning were making the data more stationarity.

    \subparagraph{Autocorrelation:}\label{autocorrelation}

    \begin{center}
    \adjustimage{max size={0.9\linewidth}{0.9\paperheight}}{Math6450_Assignment3_files/Math6450_Assignment3_39_0.png}
    \end{center}
    { \hspace*{\fill} \\}
    
    The lags show that at about lag 8, we start to see a reverse of
direction in correlation. This must mean that as it looks back further
in the year, since the weather changes this also changes AQI.

The PACF graph starts high, and after 1 lag has a steep drop off and
then tails off with no clear pattern. This can be interpreted as after 1
lag, there isn't a direct strong influence on the weekly AQI values.

    \begin{center}
    \adjustimage{max size={0.9\linewidth}{0.9\paperheight}}{Math6450_Assignment3_files/Math6450_Assignment3_41_0.png}
    \end{center}
    { \hspace*{\fill} \\}
    
    The monthly ACF graphs show a similar pattern to the weekly ACF graphs.
There is usually a seasonal pattern that can be explained by the
changing of the seasons showing that there is a correlation effect on
AQI from the seasonal changes.

The monthly PACF graph also shows similar results to the weekly PACF
plot results because of the immediate steep drop off on lag 1, and then
there is no clear pattern. This confirms the verdict above that there is
no strong and direct influence from earlier months once the 1st month is
accounted for.

    \subparagraph{Correlation Coefficients:}\label{correlation-coefficients}

    \begin{center}
    \adjustimage{max size={0.9\linewidth}{0.9\paperheight}}{Math6450_Assignment3_files/Math6450_Assignment3_44_0.png}
    \end{center}
    { \hspace*{\fill} \\}
    
    \begin{center}
    \adjustimage{max size={0.9\linewidth}{0.9\paperheight}}{Math6450_Assignment3_files/Math6450_Assignment3_44_1.png}
    \end{center}
    { \hspace*{\fill} \\}
    
    \begin{center}
    \adjustimage{max size={0.9\linewidth}{0.9\paperheight}}{Math6450_Assignment3_files/Math6450_Assignment3_44_2.png}
    \end{center}
    { \hspace*{\fill} \\}
    
    \begin{center}
    \adjustimage{max size={0.9\linewidth}{0.9\paperheight}}{Math6450_Assignment3_files/Math6450_Assignment3_44_3.png}
    \end{center}
    { \hspace*{\fill} \\}
    
    \begin{center}
    \adjustimage{max size={0.9\linewidth}{0.9\paperheight}}{Math6450_Assignment3_files/Math6450_Assignment3_44_4.png}
    \end{center}
    { \hspace*{\fill} \\}
    
    The weekly correlation shows seasonality due to what I'd assume literal
seasons, like fall, winter, summer, and spring. It makes sense that on a
lag of 1, there is a positive correlation because it is still within the
season. However, looking at lags 12, and 26, that is comparing data to a
different season which causes a negative correlation reversing the
correlation. I can say since I start this lag in January, 26 weeks ago
from January is in the summer, June or July, so it makes sense the lag
would calculate a negative correlation.

Looking at the monthly data pretty much confirms this as well, since
looking at lag 6 shows a negative correlation with the current value
meaning winter is colder than summer. So, we can confirm that there is
correlation and seasonality in this data, and we will now be able to
better select a model to forecast the AQI.

Comparing the pearson\_r scores with the plotted ACF graph shows they
both agree with each other reinforcing that correlation and seasonality
is present in this data due to changing seasons.

    \subparagraph{Data Splitting:}\label{data-splitting}

    \begin{center}
    \adjustimage{max size={0.9\linewidth}{0.9\paperheight}}{Math6450_Assignment3_files/Math6450_Assignment3_47_0.png}
    \end{center}
    { \hspace*{\fill} \\}
    
    \subparagraph{Model Selection:}\label{model-selection}

    Weekly:

SARIMA(1,0,0)(1,1,0){[}52{]} - This is also quite good.

SARIMA(1,1,1)(1,1,1){[}52{]} - might have been the best weekly
predictions

SARIMA(2,1,1)(1,1,1){[}52{]} or SARIMA(1,1,2)(1,1,1){[}52{]} - try
adding another AR or MA term \textbar{} \textbar\textbar{} Did alright
This one might have been for forecasting btbh. This one, or the simpler
(1,1,1)(1,1,1,{[}52{]})\ldots.

SARIMA(1,1,1)(2,1,1){[}52{]} or SARIMA(1,1,1)(1,1,2){[}52{]} - explore
richer seasonal structure CURRENT \textbar\textbar{} this did alright,
kind of bad tbh

SARIMA(0,1,1)(0,1,1){[}52{]} - simpler model (sometimes less is more)
\textbar{} equally as bad nearly

Monthly:

SARIMA(2,0,0)(1,0,0){[}12{]} - add another non-seasonal AR lag
\textbar{} BAD

SARIMA(1,0,1)(1,0,0){[}12{]} - add MA component \textbar{}
\textbar\textbar{} The normal monthly was bad, but the Monthly LOG was
good. Best for LOG

SARIMA(1,0,0)(2,0,0){[}12{]} or SARIMA(1,0,0)(1,0,1){[}12{]} - richer
seasonal terms \textbar{} CURRENT \textbar\textbar{} awful. infinite
upwards bye bye.

SARIMA(1,1,1)(1,1,1){[}12{]} - Log did ight, but the non-log was crap.
Best for non log.

    I tested different p, d, q, P, D, and Q values for the SARIMAX model,
and made decisions based on the ACF/PACF plots and how stationary my
data was. What I found was that taking the difference and seasonal
difference impacted all the models except for the logged aqi positively.
Also, setting p/P and q/Q to a value of 1 also showed better forecasting
and results. These values set to 1 helped the model capture short-term
autocerrlation which we saw a little bit of in the ACF plots, and the
seasonal trends were apparent from graphed data so both p/P and q/Q
helped the model perform well.

    \subparagraph{Model Parameters and
Diagnostics:}\label{model-parameters-and-diagnostics}

    
    \begin{Verbatim}[commandchars=\\\{\}]
<Figure size 1400x800 with 0 Axes>
    \end{Verbatim}

    
    \begin{center}
    \adjustimage{max size={0.9\linewidth}{0.9\paperheight}}{Math6450_Assignment3_files/Math6450_Assignment3_53_1.png}
    \end{center}
    { \hspace*{\fill} \\}
    
    
    \begin{Verbatim}[commandchars=\\\{\}]
<Figure size 1400x800 with 0 Axes>
    \end{Verbatim}

    
    \begin{center}
    \adjustimage{max size={0.9\linewidth}{0.9\paperheight}}{Math6450_Assignment3_files/Math6450_Assignment3_53_3.png}
    \end{center}
    { \hspace*{\fill} \\}
    
    
    \begin{Verbatim}[commandchars=\\\{\}]
<Figure size 1400x800 with 0 Axes>
    \end{Verbatim}

    
    \begin{center}
    \adjustimage{max size={0.9\linewidth}{0.9\paperheight}}{Math6450_Assignment3_files/Math6450_Assignment3_53_5.png}
    \end{center}
    { \hspace*{\fill} \\}
    
    \begin{center}
    \adjustimage{max size={0.9\linewidth}{0.9\paperheight}}{Math6450_Assignment3_files/Math6450_Assignment3_53_6.png}
    \end{center}
    { \hspace*{\fill} \\}
    
    \subparagraph{Forecasting:}\label{forecasting}

    \begin{center}
    \adjustimage{max size={0.9\linewidth}{0.9\paperheight}}{Math6450_Assignment3_files/Math6450_Assignment3_55_0.png}
    \end{center}
    { \hspace*{\fill} \\}
    
    \subparagraph{Model Performance:}\label{model-performance}

    \begin{center}
    \adjustimage{max size={0.9\linewidth}{0.9\paperheight}}{Math6450_Assignment3_files/Math6450_Assignment3_57_0.png}
    \end{center}
    { \hspace*{\fill} \\}
    
    \subparagraph{Residual Analysis}\label{residual-analysis}

    \begin{center}
    \adjustimage{max size={0.9\linewidth}{0.9\paperheight}}{Math6450_Assignment3_files/Math6450_Assignment3_59_0.png}
    \end{center}
    { \hspace*{\fill} \\}
    
    \subparagraph{Final Model Equation:}\label{final-model-equation}

    \begin{Verbatim}[commandchars=\\\{\}]

Weekly AQI:
  SARIMA(1, 1, 1) x (1, 1, 1, 52)
  Notation: SARIMA(p=1, d=1, q=1) x (P=1, D=1, Q=1, s=52)
  Model: (1 - φ₁B)(1 - Φ₁B\^{}52)(1 - B)(1 - B\^{}52)Yₜ = (1 + θ₁B)(1 + Θ₁B\^{}52)εₜ

Weekly Log AQI:
  SARIMA(1, 1, 1) x (1, 1, 1, 52)
  Notation: SARIMA(p=1, d=1, q=1) x (P=1, D=1, Q=1, s=52)
  Model: (1 - φ₁B)(1 - Φ₁B\^{}52)(1 - B)(1 - B\^{}52)Yₜ = (1 + θ₁B)(1 + Θ₁B\^{}52)εₜ

Monthly AQI:
  SARIMA(1, 1, 1) x (1, 1, 1, 12)
  Notation: SARIMA(p=1, d=1, q=1) x (P=1, D=1, Q=1, s=12)
  Model: (1 - φ₁B)(1 - Φ₁B\^{}12)(1 - B)(1 - B\^{}12)Yₜ = (1 + θ₁B)(1 + Θ₁B\^{}12)εₜ

Monthly Log AQI:
  SARIMA(1, 0, 1) x (1, 0, 0, 12)
  Notation: SARIMA(p=1, d=0, q=1) x (P=1, D=0, Q=0, s=12)
  Model: (1 - φ₁B)(1 - Φ₁B\^{}12)(1 - B)(1 - B\^{}12)Yₜ = (1 + θ₁B)(1 + Θ₁B\^{}12)εₜ

Best models by different criteria:
  Lowest RMSE: Monthly AQI (RMSE: 6.0475)
  Lowest MAE: Monthly AQI (MAE: 5.0463)
  Lowest MAPE: Monthly AQI (MAPE: 9.56\%)
  Lowest AIC: Monthly AQI (AIC: 1.46)
    \end{Verbatim}

    \subparagraph{Conclusion:}\label{conclusion}


    % Add a bibliography block to the postdoc
    
    
    
\end{document}
